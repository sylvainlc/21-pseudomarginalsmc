\documentclass[bj]{imsart}
%\documentclass[preprint]{imsart}



%% Packages
\RequirePackage{amsthm,amsmath,amsfonts,amssymb}
\RequirePackage[numbers]{natbib}
\RequirePackage[colorlinks,citecolor=blue,urlcolor=blue]{hyperref}
\RequirePackage{graphicx}




\usepackage[linesnumbered,vlined]{algorithm2e}
\usepackage{bbm}
\usepackage{color}
\usepackage{enumitem}
%\usepackage{graphicx}
\usepackage{ifthen}
\usepackage{mathtools}  
%\usepackage{mnsymbol}
\usepackage{stmaryrd}
\usepackage{tikz}
\usetikzlibrary{fit,positioning}
\usepackage{xargs}


\startlocaldefs
%%%%%%%%%%%%%%%%%%%%%%%%%%%%%%%%%%%%%%%%%%%%%%
%%                                          %%
%% Uncomment next line to change            %%
%% the type of equation numbering           %%
%%                                          %%
%%%%%%%%%%%%%%%%%%%%%%%%%%%%%%%%%%%%%%%%%%%%%%
\numberwithin{equation}{section}
%%%%%%%%%%%%%%%%%%%%%%%%%%%%%%%%%%%%%%%%%%%%%%
%%                                          %%
%% For Axiom, Claim, Corollary, Hypothezis, %%
%% Lemma, Theorem, Proposition              %%
%% use \theoremstyle{plain}                 %%
%%                                          %%
%%%%%%%%%%%%%%%%%%%%%%%%%%%%%%%%%%%%%%%%%%%%%%
%\theoremstyle{plain}
\newtheorem{axiom}{Axiom}
\newtheorem{claim}[axiom]{Claim}
\newtheorem{theorem}{Theorem}[section]
\newtheorem{lemma}[theorem]{Lemma}
\newtheorem{proposition}[theorem]{Proposition}
\newtheorem{corollary}[theorem]{Corollary}
%%%%%%%%%%%%%%%%%%%%%%%%%%%%%%%%%%%%%%%%%%%%%%
%%                                          %%
%% For Assumption, Definition, Example,     %%
%% Notation, Property, Remark, Fact         %%
%% use \theoremstyle{remark}                %%
%%                                          %%
%%%%%%%%%%%%%%%%%%%%%%%%%%%%%%%%%%%%%%%%%%%%%%
\theoremstyle{remark}
\newtheorem{definition}[theorem]{Definition}
\newtheorem{remark}[theorem]{Remark}
\newtheorem{example}{Example}
\newtheorem*{fact}{Fact}
%%%%%%%%%%%%%%%%%%%%%%%%%%%%%%%%%%%%%%%%%%%%%%
%% Please put your definitions here:        %%

%% Commands

\newcommandx{\accprob}[1][1=]{
\ifthenelse{\equal{#1}{}}{\alpha^{\mbox{\footnotesize{\textsf{R}}}}}{\alpha^{\mbox{\footnotesize{\textsf{MH}}}}}
}
%\newcommandx{\accprob}[1][1=]{
%\ifthenelse{\equal{#1}{}}{\alpha^{\mbox{\tiny{\textsf{AR}}}}}{\alpha^{\mbox{\tiny{\textsf{MH}}}}}
%}
\newcommandx{\accprobext}[1][1=]{
\ifthenelse{\equal{#1}{}}{\bar{\alpha}^{\mbox{\footnotesize{\textsf{R}}}}}{\bar{\alpha}^{\mbox{\footnotesize{\textsf{MH}}}}}
}
\newcommandx{\addf}[2][1=]{
\ifthenelse{\equal{#1}{}}{\termletter_{#2}}{\bar{h}_{#2 | #1}}
}
%\newcommandx{\addf}[2][1=]{
%\ifthenelse{\equal{#1}{}}{\termletter_{#2}}{\bar{h}_{#1, #2}}
%}
\newcommand{\adjfuncforward}[1]{\vartheta_{#1}}
\newcommand{\bi}[3]{J_{#1}^{(#2, #3)}}
\newcommandx{\bkmod}[2][1=]{ 
\ifthenelse{\equal{#1}{}}
{\kernel{B}_{#2}^\precpar}
{\kernel{B}_{#2}^\precpar}
}
\newcommandx{\bkw}[2][1=]{ 
\ifthenelse{\equal{#1}{}}
{\kernel{B}_{#2}}
{\kernel{B}_{#2}}
}
\newcommand{\bmf}[1]{\mathsf{F}(#1)}
%\newcommand{\bmaf}[1]{\mathsf{S}(#1)}
\newcommand{\bmaf}[1]{\mathsf{A}(#1)}

\newcommand{\calF}[2]{\mathcal{F}_{#1}^{#2}}
\newcommand{\calG}[2]{\mathcal{G}_{#1}^{#2}}
\newcommand{\cat}{\mathsf{cat}}
\newcommand{\cbound}{c(\sigma_\pm)}
\newcommand{\cboundtd}{\tilde{c}(\sigma_\pm)}
\newcommand{\cev}[1]{\reflectbox{\ensuremath{\vec{\reflectbox{\ensuremath{#1}}}}}}
\newcommand{\cond}{\mid}
\newcommand{\dbound}{d(\sigma_\pm)}
%\newcommand{\dlim}{\underset{N\to\infty}{\overset{\mathcal{D}}{\longrightarrow}}}
\newcommand{\dlim}{\overset{\mathcal{D}}{\longrightarrow}}
\newcommand{\eg}{\emph{e.g.}}
\newcommand{\epart}[2]{\ensuremath{\xi_{#1}^{#2}}}
\newcommand{\eqdef}{\coloneqq}
\newcommand{\ewght}[2]{\ensuremath{\omega_{#1}^{#2}}}
\newcommand{\fk}[2]{\mathbf{F}_{#1 | #2}}
\newcommand{\ftd}[1]{\tilde{f}_{#1}}
\newcommand{\hd}[1]{q_{#1}} 
\newcommand{\hk}{\kernel{Q}}
\newcommand{\ie}{\emph{i.e.}}
\newcommand{\im}{\operatorname{i}}
\newcommand{\incrementalvar}{\delta}
\newcommand{\ind}[2]{I_{#1}^{#2}}
\newcommand{\init}{\nu}
%\newcommand{\initwgtfunc}{\frac{\rmd \chi}{\rmd \nu}}
%\newcommand{\initwgtfunc}{\rmd \chi / \rmd \nu}
\newcommand{\initwgtfunc}{w_{-1}}
\newcommand{\intvect}[2]{\llbracket #1, #2 \rrbracket}
\newcommandx{\K}[1][1=]{\ifthenelse{\equal{#1}{}}{{\kletter}}{{M^{#1}}}}
\newcommand{\kernel}[1]{\mathbf{#1}}
\newcommand{\kissforward}[3][]
{\ifthenelse{\equal{#1}{}}{p_{#2}}
{\ifthenelse{\equal{#1}{fully}}{p^{\star}_{#2}}
{\ifthenelse{\equal{#1}{smooth}}{\tilde{r}_{#2}}{\mathrm{erreur}}}}}
\newcommand{\kletter}{M}
\newcommand{\llh}[1]{L_{#1}}
\newcommand{\lwd}[1]{l_{#1}}
\newcommand{\meas}[1]{\mathsf{M}(#1)}
\newcommand{\md}[1]{g_{#1}}
\newcommand{\mdlow}{\tau_-}
\newcommand{\mdup}{\tau_+}

\newcommand{\mk}{\kernel{G}}
\newcommand{\N}{N}
\newcommand{\Gaussian}{\mathrm{N}}
%\newcommand{\Gaussian}{\mathcal{N}}
\newcommand{\noshift}{\shiftsymbol^{\precpar}}
\newcommand{\noteJO}[1]{{\bf \textcolor{magenta}{\underline{Note JO:} #1\\}}}
\newcommand{\nset}{\mathbb{N}}
\newcommand{\nsetpos}{\mathbb{N}_{> 0}}
\newcommand{\1}{\mathbbm{1}} 
\newcommand{\partmixt}{\pi}
\newcommand{\parvec}{\theta}
\newcommand{\parspace}{\Theta}
\newcommand{\pE}{\mathbb{E}}
\newcommand{\pP}{\mathbb{P}}
\newcommand{\pplim}{\stackrel{\pP}{\longrightarrow}}
\newcommandx{\post}[2][1=]{
\ifthenelse{\equal{#1}{}}
	{\phi_{#2}}
	{\phi_{#2}^\N}
}
\newcommandx{\postmod}[2][1=]{
\ifthenelse{\equal{#1}{}}
	{\phi_{#2}^\precpar}
	{\phi_{#2}^\N}
}
%\newcommand{\powerset}[1]{2^{#1}}
\newcommand{\powerset}[1]{\mathcal{P}(#1)}
\newcommand{\precpar}{\varepsilon}
%\newcommand{\precpar}{m}
\newcommand{\precparsp}{\mathcal{E}}
\newcommand{\probmeas}[1]{\mathsf{M}_1(#1)}
\newcommand{\prop}[1]{\mathbf{P}_{#1}}
\newcommand{\propdens}[1]{p_{#1}}
\newcommand{\retrok}{\boldsymbol{\mathcal{L}}}
\newcommand{\retroknorm}{\bar{\boldsymbol{\mathcal{L}}}}
\newcommand{\retrokmod}{\boldsymbol{\mathcal{L}}^\precpar}
\newcommand{\retrokmodnorm}{\bar{\boldsymbol{\mathcal{L}}}^{\precpar}}
\newcommand{\retrokmodmod}{\boldsymbol{\mathcal{R}}^\precpar}
\newcommand{\retrokmodmodnorm}{\bar{\boldsymbol{\mathcal{R}}}^\precpar}
\newcommand{\rmd}{d}
\newcommand{\rset}{\ensuremath{\mathbb{R}}}
\newcommand{\rsetnn}{\rset_{\geq 0}}
\newcommand{\rsetpos}{\rset_{> 0}}
\newcommand{\shiftbwd}{\cev{\shiftsymbol}^{\precpar}}
\newcommand{\shiftfwd}{\vec{\shiftsymbol}^{\, \precpar}}
\newcommand{\shiftsymbol}{\varphi}
\newcommand{\sumwght}[2][]{%
\ifthenelse{\equal{#1}{}}{\ensuremath{\Omega_{#2}}}{\ensuremath{\Omega_{#2}^{(#1)}}}}
\newcommand{\sumwghthat}[2][]{%
\ifthenelse{\equal{#1}{}}{\ensuremath{\widehat{\Omega}_{#2}}}{\ensuremath{\widehat{\Omega}_{#2}^{(#1)}}}}
\newcommand{\tensprod}{\varotimes}
\newcommand{\termletter}{\tilde{h}}
\newcommandx{\tstat}[2][1=]{
\ifthenelse{\equal{#1}{}}
	{\tstatletter_{#2}}
	{\tau_{#2}^{#1}}
}
\newcommand{\termprime}{\Delta_N^\prime}
\newcommand{\termbis}{\Delta_N^{\prime \prime}}
\newcommand{\tosupp}[1]{\textcolor{red}{#1}}
\newcommand{\trmletter}{\boldsymbol{\Pi}}
\newcommand{\trm}[1]{\trmletter_{#1}}
\newcommand{\trmext}[1]{\bar{\trmletter}_{#1}}
\newcommand{\tstatletter}{\kernel{T}}
\newcommandx\tstatmod[2][1=]{
\ifthenelse{\equal{#1}{}}
	{\tstatletter^{\varepsilon}_{#2}}
	{\tau^{\varepsilon}_{#2}^{#1}}
}
\newcommand{\ud}[1]{\uksymbol_{#1}} 
\newcommand{\udlow}{\sigma_-}
\newcommand{\udmod}[1]{\uksymbol^\precpar_{#1}}
\newcommand{\udup}{\sigma_+}
\newcommand{\uk}[1]{\mathbf{L}_{#1}}
\newcommand{\ukest}[2]{\uksymbol_{#1} \langle #2 \rangle}
\newcommand{\ukestvar}[1]{\varsigma_{#1}^2}
\newcommand{\ukdist}[1]{\mathbf{R}_{#1}}
\newcommand{\ukmod}[1]{\mathbf{L}_{#1}^\precpar}
\newcommand{\uksymbol}{\ell}
\newcommand{\wgtfunc}[1]{w_{#1}}
\newcommand{\wgtfuncideal}[1]{\bar{w}_{#1}}
\newcommand{\wgtfuncmod}[1]{w^{\varepsilon}_{#1}}
\newcommand{\xarb}{x^\ast}
%\newcommand{\xarb}{\hat{x}}
\newcommand{\Xfd}{\mathcal{X}}
\newcommand{\Xset}{\mathsf{X}}

\newcommand{\Yfd}{\mathcal{Y}}
\newcommand{\Yset}{\mathsf{Y}}



\newcommand{\Zfd}{\mathcal{Z}}
\newcommand{\zpart}[2]{\zeta_{#1}^{#2}} 
\newcommand{\zset}{\mathbb{Z}}
\newcommand{\Zset}{\mathsf{Z}}



%\newcommand{\mdup}{\tau_+}
%\newcommand{\mdlow}{\tau_-}
%\newcommand{\hdlow}{\sigma_-}
%\newcommand{\hdup}{\sigma_+}
\newcommand{\wgtfuncext}[1]{w_{#1}}
\newcommand{\ewghthat}[2]{\omega_{#1}^{#2}}
\newcommandx\tstathat[2][1=]{
\ifthenelse{\equal{#1}{}}
	{\tstatletter_{#2}}
	{\tau_{#2}^{#1}}
}
\newcommand{\M}{M}

%% Colors 

\definecolor{violet}{cmyk}{0.79,0.88,0,0}
\definecolor{lavander}{cmyk}{0,0.48,0,0}
\definecolor{burntblue}{cmyk}{0.86,0.30,0.18,0}
\definecolor{burntorange}{cmyk}{0,0.52,1,0}
\definecolor{burntgreen}{cmyk}{0.62,0.44,0.47,0}
\definecolor{colorproof}{RGB}{80,93,113}
\definecolor{lightr}{RGB}{204,0,0}
\definecolor{palegreen}{cmyk}{0.86,0.30,0.96,0}

%% Hypotheses

\newcounter{hypH}
\newenvironment{hypH}{\refstepcounter{hypH}\begin{itemize}
\item[({\bf H\arabic{hypH}})]}{\end{itemize}}
\newcommandx{\hypref}[2][1=]{
\ifthenelse{\equal{#1}{}}
{\hspace{-1mm}(\textbf{H\ref{#2}})\hspace{-1mm}}
{\hspace{-1mm}(\textbf{H\ref{#1}--\ref{#2}})\hspace{-1mm}}
}





%%%%%%%%%%%%%%%%%%%%%%%%%%%%%%%%%%%%%%%%%%%%%%

\endlocaldefs

\begin{document}

\begin{frontmatter}
\title{Supplement to `Pseudo-marginal PaRIS samplers'}
%\title{A sample article title with some additional note\thanksref{t1}}
\runtitle{Supplement to `Pseudo-marginal PaRIS samplers'}
%\thankstext{t1}{Working title.}

\begin{aug}
\author[A]{\fnms{Pierre} \snm{Gloaguen}\ead[label=e1]{pierre.gloaguem@agroparistech.fr}},
\author[B]{\fnms{Sylvain} \snm{Le Corff}\ead[label=e2]{sylvain-lecorff@telecom-sudparis.eu}}
\and
\author[C]{\fnms{Jimmy} \snm{Olsson}\ead[label=e3]{jimmyol@kth.se}}
%%%%%%%%%%%%%%%%%%%%%%%%%%%%%%%%%%%%%%%%%%%%%%
%% Addresses                                %%
%%%%%%%%%%%%%%%%%%%%%%%%%%%%%%%%%%%%%%%%%%%%%%
\address[A]{Department,
AgroParisTech, Paris, France.
\printead{e1}}

\address[B]{Department,
T\'el\'ecom SudParis, Paris, France.
\printead{e2}}

\address[C]{Department,
KTH Royal Institute of Technology, Stockholm, Sweden. 
\printead{e3}}

\end{aug}

\begin{abstract}
Here we will put a concise abstract.
\end{abstract}

\begin{keyword}
\kwd{First keyword}
\kwd{second keyword}
\end{keyword}

\end{frontmatter}

\section{Introduction}
\label{sec:introduction}
Let $(\Xset_n, \Xfd_n)_{n \in \nset}$ be a sequence of general state spaces and let, for all $n \in \nset$, $\uk{n} : \Xset_n \times \Xfd_{n + 1} \to \rsetnn$ be bounded kernels in the sense that $\sup_{x \in \Xset_n} \uk{n}(x, \Xset_{n + 1}) < \infty$. We will assume a dominated model where each kernel $\uk{n}$ has a kernel density $\ud{n}$ with respect to some $\sigma$-finite reference measure $\mu_{n + 1}$ on $\Xfd_{n + 1}$. Finally, let $\chi$ be some bounded measure on $\Xfd_0$. In the following, we denote state-space product sets and $\sigma$-fields by $\Xset^n \eqdef \Xset_0 \times \cdots \times \Xset_n$ and $\Xfd^n \eqdef \Xfd_0 \tensprod \cdots \tensprod \Xfd_n$, respectively, and consider probability measures  
\begin{equation} \label{eq:def:post}
\post{0:n}(\rmd x_{0:n}) \propto \chi(\rmd x_0) \prod_{m = 0}^{n - 1} \uk{m}(x_m, \rmd x_{m + 1}), \quad n \in \nset, 
\end{equation}
on these product spaces.\footnote{We will always use the standard convention $\prod_{\varnothing} = 1$, implying that $\post{0} \propto \chi$.} 

The generality of the model \eqref{eq:def:post} is striking. In the special case where each $\uk{n}$ can be decomposed as $\uk{n}(x_n, \rmd x_{n + 1}) = \md{n}(x_n) \, \hk_n(x_n, \rmd x_{n + 1})$ for some Markov kernel $\hk_n$ and some nonnegative potential function $\md{n}$, \eqref{eq:def:post} yields the \emph{Feynman-Kac path models} \cite{delmoral:2004}, which are applied in a large variety of scientific and engineering disciplines, including statistics, physics, biology, and signal processing. 
They appear naturally in the context of \emph{hidden Markov models} (HMMs), where a Markov chain $(X_n)_{n \in \nset}$ with kernels $(\hk_n)_{n \in \nset}$ and initial distribution $\chi$ is only partially observed through a sequence $(Y_n)_{n \in \nset}$ of observations being conditionally independent given the Markov states. General state-space HMMs are prevalent in time-series and sequential-data analysis, and are used extensively in, \eg, movement ecology \cite{michelot2016movehmm}, energy-consumption modeling \cite{candanedo2017methodology}, genomics \cite{yau2011bayesian}, target tracking \cite{sarkka2007rao}, enhancement and segmentation of speech and audio signals \cite{rabiner1989tutorial}; see also \cite{Cappe:2005:IHM:1088883,sarkka2013bayesian} and the numerous references therein. In the HMM context, $g_n$ plays the role of the likelihood of the state $X_n$ given the observation $Y_n$, and $\post{0:n}$ describes the \emph{joint-smoothing distribution} at time $n$, \ie, the joint posterior of the hidden states $X_0, \ldots, X_n$ given the corresponding observations; see Example~\ref{ex:state-space:models} for details. We will adopt this terminology throughout the present paper and simply refer to the distributions defined in \eqref{eq:def:post} as `smoothing distributions' also in the general case. 

Given a sequence $(\addf{n})_{n \in \nset}$ of measurable functions $\addf{n} : \Xset_n \times \Xset_{n + 1} \to \rset$, the aim of the present paper is the online approximation of expectations of \emph{additive functionals}   
\begin{equation} \label{eq:add:func}
    h_n : \Xset^n \ni x_{0:n} \mapsto \sum_{m = 0}^{n - 1} \addf{m}(x_m, x_{m + 1})
\end{equation}
under the distribution flow $(\post{0:n})_{n \in \nset}$ using \emph{sequential Monte Carlo} (SMC) methods. The problem of computing expectations of functionals of type \eqref{eq:add:func} under these distributions will be referred to as `additive smoothing'. Operating on HMMs, online additive smoothing is instrumental for, \emph{e.g.},    
\begin{enumerate}
    \item[--] \emph{path reconstruction}, \ie, the estimation of hidden states given observations. Especially in \emph{fixed-point smoothing}, where interest is in computing the expectations of $h(X_m)$ conditionally on $Y_0, \ldots, Y_n$ for some given $m$ and test function $h$ as $n$ tends to infinity, a problem that can be cast into our framework by letting, in \eqref{eq:add:func}, $\addf{m}(x_m, x_{m + 1}) = h(x_m)$ and $\addf{\ell} \equiv 0$ for all $\ell \neq m$. 
    \item[--] \emph{parameter inference}, where additive smoothing is a key ingredient in the computation of log-likelihood gradients (\emph{score functions}) via Fisher's identity or the intermediate quantity of the \emph{expectation-maximisation} (EM) \emph{algorithm}; see, \eg, \cite[Chapter~10]{Cappe:2005:IHM:1088883}. On-the-fly computation becomes especially important in online implementations via, \eg, the \emph{online EM} or \emph{recursive maximum likelihood} approaches \cite{cappe:2009,legland:mevel:1997}.  
 \end{enumerate}

As closed-form solutions to this smoothing problem can be obtained only for linear Gaussian models or models with finite state spaces $(\Xset_n)_{n \in \nset}$, loads of papers have been written over the years with the aim of developing SMC-based approximative solutions. Most of these works assume that each density $\ud{n}$ (or, in the HMM case, the transition density of $\hk_n$ and the likelihood $g_n$) is available in a closed form; however, this is not the case for a large number of interesting models, including most state-space HMMs governed by stochastic differential equations. Still, there are a few exceptions in the literature. In \cite{fearnhead2008particle} (see also \cite{fearnhead:papaspiliopoulos:roberts:stuart:2010}), the authors showed that asymptotically consistent online state estimation in partially observed diffusion processes can be achieved by means of a \emph{random-weight particle filter}, in which unavailable importance weights are replaced by unbiased estimates (produced using so-called \emph{generalized Poisson estimators} \cite{beskos:papaspiliopoulos:roberts:fearnhead:2006}). This approach is closely related to \emph{pseudo-marginal methods} \cite{andrieu:robert:2009}, since the unbiasedness allows the true, intractable target to be embedded into an extended distribution having the target as a marginal; as a consequence, the consistency of the algorithm follows straightforwardly from standard SMC convergence results. A similar pseudo-marginal SMC approach was developed in \cite{mcgree:drovandi:white:pettitt:2016} for random effects models with non-analytic likelihood. In \cite{olsson:strojby:2011}, this technology was cast into the framework of \emph{fixed-lag particle smoothing} of additive state functionals, where the well-known particle-path degeneracy of naive particle smoothers is avoided at the price of a lag-induced bias. Recently, \cite{yonekura:beskos:2020} designed a random-weight version of the forward-only particle smoother proposed in \cite{delmoral:doucet:singh:2010}, whose computational complexity is quadratic in the number $\N$ of particles, yielding a strongly consistent---though computationally demanding---algorithm. Moreover, \cite{gloaguen2018online} extended the random-weight particle filtering approach to the \emph{particle-based, rapid incremental smoother} (PaRIS), proposed in \cite{olsson:westerborn:2014b} as a means for additive smoothing in HMMs, yielding an algorithm with just linear complexity. The complexity of the latter algorithm is appealing; however, the schedule was not furnished with any theoretical results concerning the asymptotic properties and long-term stability of the estimator or the effect of the weight randomisation on the accuracy. In addition, the algorithm is restricted to partially observed diffusions and unbiased weight estimation, calling for strong assumptions on the unobserved process. 
        
In the present paper we further develop the approach in \cite{gloaguen2018online} and extend the PaRIS to online additive smoothing in general models in the form \eqref{eq:def:post} and the scenario where the transition densities $(\ud{n})_{n \in \nset}$ are intractable but can be estimated by means of simulation. These estimates may be unbiased or biased. In its original form, the PaRIS avoids particle-path degeneracy by alternating two sampling operations, one that propagates a sample of forward-filtering particles and another that resamples a set of backward-smoothing statistics, and the proposed method replaces the sampling distributions associated with these operations by suitable pseudo-marginals. This leads to an $\mathcal{O}(\N)$ algorithm that can be applied to a wide range of smoothing problems, including additive smoothing in partially observed diffusion processes and additive \emph{approximate Bayesian computation smoothing} \cite{martin:jasra:singh:whiteley:delmoral:maccoy:2014}. As illustrated by our examples, it covers the random-weight algorithms proposed in \cite{fearnhead2008particle} and \cite{gloaguen2018online} as special cases and provides, as another special case, an extension of the original PaRIS proposed in \cite{olsson:westerborn:2014b} to general path models \eqref{eq:def:post} and auxiliary particle filters. In addition, the proposed method is furnished with a rigorous theoretical analysis, the results of which can be summarised as follows. 

\begin{itemize}
\item We establish exponential concentration and asymptotic normality of the estimators produced by the algorithm. These results extend analogous results established in \cite{olsson:westerborn:2014b} for the original PaRIS (operating on fully dominated HMMs using the bootstrap particle filter), and the additional randomness of the pseudo-marginals can be shown to affect the asymptotic variance through an additional positive term. The fact that our smoothing algorithm, as explained above, involves two separate levels of pseudo-marginalisation makes this extension highly non-trivial. 
\item Under strong mixing assumptions we establish the long-term stochastic stability of our algorithm by showing that its asymptotic variance grows at most linearly in $n$. As explained in \cite[Section~1]{olsson:westerborn:2014b}, this is optimal for a path-space Monte Carlo estimator. As a by-product of this analysis, we obtain a time-uniform bound on the asymptotic variance of the random-weight particle filter. 
\item As mentioned above, we do not require the estimators of $(\uk{n})_{n \in \nset}$ to be unbiased. The bias is assumed to be regulated by some precision parameter $\precpar$ (see \hypref{assum:bias:bound}), and under additional strong mixing assumptions we establish an $\mathcal{O}(n \precpar)$ bound on the asymptotic bias of the final estimator. In addition, we obtain, as a by-product, an $\mathcal{O}(\precpar)$ bound for the random-weight particle filter. These results are the first of their kind.   
\end{itemize}

The paper is structured as follows. In Section~\ref{sec:preliminaries} we cast, under the temporary assumption that each $\ud{n}$ is tractable, the PaRIS into the general model \eqref{eq:def:post} and auxiliary particle filters and define carefully the two forward and backward sampling operations constituting the algorithm. Since this extension is of independent interest, we provide the details. In Section~\ref{sec:pseudo:marginal:PaRIS} we show how pseudo-marginal forward and backward sampling allow the temporary tractability assumption to be abandoned. Section~\ref{sec:theoretical:results} presents all theoretical results and although an extensive numerical study of the proposed scheme is beyond the scope of our paper, we present a minor numerical illustration of the $\mathcal{O}(n \precpar)$ bias bound in Section~\ref{sec:numerical:results}. All proofs are found in the supplement. 


 


\section{Preliminaries}
\label{sec:preliminaries}
We first introduce some general notation. 
For any $m, n \in \zset$ such that $m \leq n$, we let $\intvect{m}{n}$ denote the set $\{m, \ldots, n\}$. For arbitrary elements $a_\ell$, $\ell \in \intvect{m}{n}$, we denote vectors by $a_{m:n} = (a_m, \ldots, a_n)$. The sets of measures, probability measures, and real-valued bounded measurable functions on some given state-space $(\mathsf{X}, \mathcal{X})$ are denoted by $\meas{\mathcal{X}}$, $\probmeas{\mathcal{X}}$, and $\bmf{\mathcal{X}}$, respectively. For any measure $\mu$ and measurable function $h$ we let $\mu h \eqdef \int h(x) \, \mu(\rmd x)$ denote the Lebesgue integral of $h$ with respect to $\mu$ whenever this is well defined. When $h$ depends on several variables we may write $\mu[h(\cdot,y)]\eqdef \int h(x,y) \, \mu(\rmd x)$ to avoid any confusion. We will write $\mu^2 f = (\mu f)^2$ (whereas $\mu f^2 = \mu(f^2)$). For any finite set $S$, $\powerset{S}$ denotes the power set of $S$. The following kernel notation will be used repeatedly in the paper. Let $(\mathsf{X}, \mathcal{X})$ and $(\mathsf{Y}, \mathcal{Y})$ be general state spaces and $\kernel{K} : \mathsf{X} \times \mathcal{Y} \to \rsetnn$ some transition kernel. Then $\kernel{K}$ induces two operators, one acting on measurable functions and the other on measures. More precisely, for any $h \in \bmf{\mathcal{X} \tensprod \mathcal{Y}}$ and $\mu \in \meas{\mathcal{X}}$, let 
$$
\kernel{K} h : \mathsf{X} \ni x \mapsto \int h(x, y) \, \kernel{K}(x, \rmd y), \quad \mu \kernel{K} : \mathcal{Y} \ni A \mapsto \int \mu(\rmd x) \, \kernel{K}(x, A)
$$ 
Moreover, let $(\mathsf{Z}, \mathcal{Z})$ be a third state space and $\kernel{K}' : \mathsf{Y} \times \mathcal{Z} \to \rsetnn$ another kernel; then the product of $\kernel{K}$ and $\kernel{K}'$ is the kernel defined by 
$$
\kernel{K} \kernel{K}' : (x, A) \ni \mathsf{X} \times \mathcal{Z} \mapsto \int \kernel{K}(x, \rmd y) \, \kernel{K}'(y, A). 
$$    

\subsection{Model and aim}
\label{sec:model}

With notations as in Section~\ref{sec:introduction}, define, for each $n \in \nset$ and $m \in \intvect{0}{n}$, the kernel 
\begin{equation} \label{eq:def:uk:products}
    \uk{m, n}(x_{0:m}', \rmd x_{0:n + 1}) \eqdef \delta_{x_{0:m}'}(\rmd x_{0:m}) \prod_{\ell = m}^n \uk{\ell}(x_\ell, \rmd x_{\ell + 1}) 
\end{equation}
on $\Xset^m \times \Xfd^{n + 1}$. In addition, let $\uk{n, n - 1} = \operatorname{id}$. Note that $\uk{n, n}$ is different from $\uk{n}$ in the sense that the former is defined on $\Xset^n \times \Xfd^{n + 1}$ whereas the latter is defined on $\Xset_n \times \Xfd_{n + 1}$. We will always assume that for all $n \in \nset$, $\chi \uk{0, n - 1} \1_{\Xset^n} = \chi \uk{0} \cdots \uk{n - 1} \1_{\Xset_n} > 0$. Since each mapping $\uk{m, n} \1_{\Xset^n}$ depends only on the last coordinate $x_m$, a version of this mapping with domain $\Xset_m$ is well defined; we will denote the latter by the same symbol and write $\uk{m, n} \1_{\Xset^n}(x_m)$, $x_m \in \Xset_m$, when needed. Using the previous notations, the path measures \eqref{eq:def:post} can be expressed as 
\begin{equation}
\label{eq:FK:path}
    \post{0:n}(\rmd x_{0:n}) = \frac{\chi \uk{0, n - 1}(\rmd x_{0:n})}{\chi \uk{0, n - 1} \1_{\Xset^n}}, \quad n \in \nset. 
\end{equation} 
For each $n \in \nset$, let  
$ 
\post{n} : A \ni \Xfd_n \mapsto \post{0:n}(\Xset^{n - 1} \times A)
$
denote the marginal of $\post{0:n}$ with respect to the last component. In the case of state-space models, see Example~\ref{ex:state-space:models}, $\post{0:n}$ is usually referred to as the joint smoothing distribution and $\post{n}$ as the filtering distribution. Note that the path- and marginal-measure flows can be expressed recursively as  
\begin{equation} \label{eq:recursion:FK:path}
    \post{0:n + 1}(\rmd x_{0:n + 1}) = \frac{\post{0:n} \uk{n, n}(\rmd x_{0:n + 1})}{\post{0:n} \uk{n, n} \1_{\Xset^{n + 1}}} = \frac{\post{0:n} \uk{n, n}(\rmd x_{0:n + 1})}{\post{n} \uk{n} \1_{\Xset_{n + 1}}}, \quad n \in \nset, 
\end{equation}
and 
\begin{equation} \label{eq:recursion:FK:marg}
    \post{n + 1}(\rmd x_{n + 1}) = \frac{\post{n} \uk{n}(\rmd x_{n + 1})}{\post{n} \uk{n} \1_{\Xset_{n + 1}}}, \quad n \in \nset, 
\end{equation}
respectively. Given some sequence $(\addf{n})_{n \in \nset}$ of functions $\addf{n} : \Xset_n \times \Xset_{n + 1} \to \rset$, our aim is, as declared in Section~\ref{sec:introduction}, the online approximation of $(\post{0:n} h_n)_{n \in \nset}$, where $h_n$ defined by \eqref{eq:add:func}. 

\begin{remark} \label{we:vs:delmoral}
    Note that our framework is equivalent to the Feynman-Kac models considered in \cite[Section~1.3]{delmoral:2004}, where it is assumed that each kernel $\uk{n}$ can be decomposed into a Markov transition kernel $\kernel{M}_n$ on $\Xset_n \times \Xfd_{n + 1}$ and a potential function $\md{n} \in \bmf{\Xfd_n}$ according to $\uk{n}(x_n, \rmd x_{n + 1}) = \md{n}(x_n) \, \kernel{M}(x_n, \rmd x_{n + 1})$. Indeed, as soon as $\uk{n}$ is bounded, such a decomposition is always possible by letting $\kernel{M}(x_n, \rmd x_{n + 1}) \eqdef \uk{n}(x_n, \rmd x_{n + 1}) / \uk{n}(x_n, \Xset_{n + 1})$ and $\md{n}(x_n) \eqdef \uk{n}(x_n, \Xset_{n + 1})$. However, in our case this potential function is, contrary to what is assumed in \cite{delmoral:2004}, generally intractable, since $\uk{n}(x_n, \Xset_{n + 1})$ is typically unknown for the models that we will consider. Moreover, as noted in \cite[Section~1.3]{delmoral:2004}, the path model $(\post{0:n})_{n \in \nset}$ and the marginal model $(\post{n})_{n \in \nset}$ have the same mathematical structure in the sense that the path model can be formulated as a marginal model evolving on the spaces $(\Xset_n', \Xfd_n')_{n \in \nset}$, where $\Xset_n' \eqdef \Xset^n$ and $\Xfd_n' \eqdef \Xfd^n$, according to the initial distribution $\chi \eqdef \chi'$ and the transition kernels $(\uk{n}')_{n \in \nset}$, where $\uk{n}' \eqdef \uk{n, n}$. Still, the kernels $(\uk{n}')_{n \in \nset}$ involve transitions according to Dirac measures, which makes the model formed by $\chi'$ and $(\uk{n}')_{n \in \nset}$ ill-suited for naive particle approximation; see Section~\ref{sec:SMC} for further discussion. 
\end{remark}

\begin{example}[state-space models]
\label{ex:state-space:models}
Let $(\Xset, \Xfd)$ and $(\Yset, \Yfd)$ be general state spaces and $(\hk_n)_{n \in \nset}$ and $(\mk_n)_{n \in \nset}$ sequences of Markov kernels on $\Xset \times \Xfd$ and $\Xset^2 \times \Yfd$, respectively. In addition, let $\chi$ be some probability measure on $\Xfd$. Consider a fully dominated model where all $\mk_n$ and $\hk_n$ have transition densities $\md{n}$ and $\hd{n}$ with respect to some reference measures $\nu$ and $\mu$ on $\Yfd$ and $\Xfd$, respectively. Let $\{X_0, (X_n, Y_n) : n \in \nsetpos\}$ be the canonical Markov chain induced by the initial distribution $\chi$ and the Markov transition kernel $\hk_n(x_n, \rmd x_{n + 1}) \, \mk_n(x_n, x_{n + 1}, \rmd y_{n + 1})$ (which has no dependence on the $y_n$ variable; the same dynamics hence applies to the first transition $X_0 \rightsquigarrow (X_1, Y_1)$) and denote by $\pP_\chi$ its law with corresponding expectation $\pE_\chi$. In this model, we assume that the \emph{state process} $(X_n)_{n \in \nset}$ is latent and only partially observed thought the \emph{observation process} $(Y_n)_{n \in \nsetpos}$. It can be shown that (i) the state process is itself a Markov chain with initial distribution $\chi$ and transition kernels $(\hk_n)_{n \in \nset}$ and (ii) conditionally on the state process, the observations are independent and such that the marginal distribution of $Y_n$ is given by $\mk_{n - 1}(X_{n - 1}, X_n, \cdot)$ for all $n$. In the case where $\mk_{n - 1}$ does not depend on $x_{n - 1}$, the model is a fully adapted general state-space HMM; see \cite[Section~2.2]{Cappe:2005:IHM:1088883}. In this setting, the joint-smoothing distribution at time $n \in \nset$ is, for a given record $y_{1:n} \in \Yset^n$ of observations, defined as the probability measure  
\begin{equation}
\label{eq:smooth}
    \post{0:n} \langle y_{1:n} \rangle (\rmd x_{0:n}) \eqdef \llh{n}^{-1}(y_{1:n}) \chi(\rmd x_0) \prod_{m = 0}^{n - 1} \hk_m(x_m, \rmd x_{m + 1}) \, \md{m}(x_m, x_{m + 1}, y_{m + 1}), 
\end{equation}
on $\Xfd^n$, where   
\begin{equation}
    \label{eq:likelihood}
    \llh{n}(y_{1:n}) \eqdef \idotsint \chi(\rmd x_0) \prod_{m = 0}^{n - 1} \hk_m(x_m, \rmd x_{m + 1}) \, \md{m}(x_m, x_{m + 1}; y_{m + 1})
\end{equation}
is the observed data likelihood. Along the lines of \cite[Proposition~3.1.4]{Cappe:2005:IHM:1088883} one may show that $\post{0:n} \langle Y_{1:n} \rangle$ is, $\pP_\chi$-a.s., the conditional distribution of $X_{0:n}$ given $Y_{1:n}$. The marginal  $\post{n} \langle y_{1:n} \rangle$ of the joint smoothing distribution with respect to its last component $x_n$ is referred to as the \emph{filtering distribution at time $n$}. Consequently, by defining kernel densities $\ud{n}(x_n, x_{n + 1}) \eqdef \md{n}(x_n, x_{n + 1}, y_{n + 1}) \hd{n}(x_n, x_{n + 1})$ for all $n \in \nset$ (while keeping dependence on observations implicit) and letting $\uk{n}$ be the induced transition kernels, the joint-smoothing distributions may be expressed in the form \eqref{eq:FK:path}.
\end{example}

The measures \eqref{eq:FK:path} are generally intractable in two ways. First, in many applications the transition densities $(\ud{n})_{n \in \nset}$ cannot be evaluated pointwise. Returning to Example~\ref{ex:state-space:models} and the context of smoothing in state-space models, this is typically the case when the dynamics of the latent process is governed by a stochastic differential equation (see Example~\ref{eq:durham:gallant} below). Second, even in the case where these transition densities are evaluable, the normalising constant in \eqref{eq:FK:path} is generally intractable. Thus, in order to solve the smoothing problem in full generality, one needs to be able to handle this double intractability, which is the goal of the algorithm that we will develop next. We will proceed in two steps. In the next section, Section~\ref{sec:PaRIS}, we will solve the additive smoothing problem under the temporary assumption that the densities $(\ud{n})_{n \in \nset}$ are tractable; the resulting extension of the PaRIS proposed in  \cite{olsson:westerborn:2014b} to general models in the form \eqref{eq:def:post} and auxiliary particle filters is of independent interest. Then, in Section~\ref{sec:pseudo:marginal:PaRIS}, we abandon the temporary assumption of tractability and assume that the user has only possibly biased proxies of these densities at hand. 

\subsection{The PaRIS}
\label{sec:PaRIS}

%%%%%%%
%% The APF
%%%%%%%

\subsubsection{Auxiliary particle filters}
\label{sec:SMC}

Assume for a moment that each transition density $\ud{n}$ is available in a closed form. Then standard SMC methods (see \cite{chopin:papaspiliopoulos:2020} for a recent introduction) can be used to approximate the distribution flows $(\post{0:n})_{n \in \nset}$ and $(\post{n})_{n \in \nset}$ using Monte Carlo samples generated recursively by means of sequential importance sampling and resampling operations. In order to set notations, let us recall the most general class of such algorithms, the so-called auxiliary particle filters \cite{pitt:shephard:1999}. In the light of Remark~\ref{we:vs:delmoral} it is enough to consider particle approximation of the marginals $(\post{n})_{n \in \nset}$. We proceed recursively and assume that we, at time $n \in \nset$, have at hand a sample $(\epart{n}{i}, \ewght{n}{i})_{i = 1}^\N$ of $\Xset_n$-valued \emph{particles} (the $\epart{n}{i}$) and associated nonnegative importance weights (the $\ewght{n}{i}$) such that the self-normalised estimator $\post[\N]{n} h \eqdef \sumwght{n}^{-1} \sum_{i = 1}^\N \ewght{n}{i} h(\epart{n}{i})$, with $\sumwght{n} \eqdef \sum_{i = 1}^\N \ewght{n}{i}$, approximates $\post{n} h$ for every $\post{n}$-integrable function $h$. Then plugging $\post[\N]{n}$ into the recursion \eqref{eq:recursion:FK:marg} yields the approximation $\sum_{i = 1}^\N \pi_n(i, \rmd x)$ of $\post{n + 1}$, where 
\begin{equation} \label{eq:def:partmixt}
\partmixt_n(i, \rmd x) \propto \ewght{n}{i} \uk{n}(\epart{n}{i}, \rmd x)
\end{equation}
is a mixture distribution on $\powerset{\intvect{1}{\N}} \tensprod \Xfd_{n + 1}$. In order to form new particles approximating $\post{n + 1}$, we may draw, using importance sampling, pairs $(\ind{n + 1}{i}, \epart{n + 1}{i})_{i = 1}^\N$ of indices and particles from $\partmixt_n$ and discard the former. For this purpose, we introduce some instrumental mixture distribution 
\begin{equation} \label{eq:cond:instrumental:mixture}
\rho_n(i, \rmd x) \propto \ewght{n}{i} \adjfuncforward{n}(\epart{n}{i}) \, \prop{n}(\epart{n}{i}, \rmd x)
\end{equation}
on the same space, where $\adjfuncforward{n}$ is a real-valued positive \emph{adjustment-weight function} on $\Xset_n$ and $\prop{n}$ is a proposal kernel on $\Xset_n \times \Xfd_{n + 1}$ such that $\uk{n}(x, \cdot) \ll \prop{n}(x, \cdot)$ for all $x \in \Xset_n$. Adjustment multiplier weights were introduced in \cite{pitt:shephard:1999} to propose data-driven proposal distributions and design robust filtering estimates. These weights usually depend on the new observation and are often chosen to approximate the predictive likelihood of the new observation given the current particle. For further discussions see for instance \cite[Chapter~8]{Cappe:2005:IHM:1088883}.

We will always assume that $\prop{n}$ has a transition density $\propdens{n}$ with respect to $\mu_{n + 1}$. A draw $(\ind{n + 1}{i}, \epart{n + 1}{i})$ from $\rho_n$ is easily generated by first drawing $\ind{n + 1}{i}$ from the categorical distribution induced by the adjusted importance weights and then drawing $\epart{n + 1}{i}$ by randomly moving the selected ancestor $\epart{n}{{\ind{n + 1}{i}}}$ according to the proposal kernel. These steps are often referred to as \emph{selection} and \emph{mutation}, respectively. 
Finally, each draw $\epart{n + 1}{i}$ is assigned the updated importance weight $\ewght{n + 1}{i}$ proportional to $\rmd \partmixt_n / \rmd \rho_n (\ind{n + 1}{i}, \epart{n + 1}{i})$, and the estimator $\post[\N]{n + 1} h = \sumwght{n + 1}^{-1} \sum_{i = 1}^\N \ewght{n + 1}{i} h(\epart{n + 1}{i})$ approximates $\post{n + 1} h$ for every $\post{n + 1}$-integrable $h$. The full update, which we will refer to as \emph{forward sampling} and express in a short form as
\begin{equation} \label{eq:forward:sampling}
    (\epart{n + 1}{i}, \ewght{n + 1}{i})_{i = 1}^\N \sim \mathsf{FS}((\epart{n}{i}, \ewght{n}{i})_{i = 1}^\N),   
\end{equation}
is summarised in Algorithm~\ref{alg:ideal:SMC}.\footnote{Mathematically, the forward sampling operation defines a Markov transition kernel, which motivates the use of the symbol $\sim$ in \eqref{eq:forward:sampling}.} 

\begin{algorithm}[h] 
    \KwData{$(\epart{n}{i}, \ewght{n}{i})_{i = 1}^\N$}
    \KwResult{$(\epart{n + 1}{i}, \ewght{n + 1}{i})_{i = 1}^\N$}
    \For {$i = 1 \to \N$}{
        draw $\ind{n + 1}{i} \sim \cat(\{ \adjfuncforward{n}(\epart{n}{j}) \ewght{n}{j} \}_{j = 1}^\N)$\;
        draw $\epart{n + 1}{i} \sim \prop{n}(\epart{n}{{\ind{n + 1}{i}}}, \cdot)$\;
        set $\ewght{n + 1}{i} \gets \frac{\ud{n}(\epart{n}{{\ind{n + 1}{i}}}, \epart{n + 1}{i})}{\adjfuncforward{n}(\epart{n}{{\ind{n + 1}{i}}}) \propdens{n}(\epart{n}{{\ind{n + 1}{i}}}, \epart{n + 1}{i})}$\;
}
\caption{Forward sampling, \textsf{FS}} \label{alg:ideal:SMC}
\end{algorithm}

With this terminology, the auxiliary particle filter consists of iterated forward sampling operations, and we will assume that the process is initialised by sampling independent particles $(\epart{0}{i})_{i = 1}^\N$ from some proposal $\init$ on $(\Xset_0, \Xfd_0)$ such that $\chi \ll \init$ and letting $\ewght{0}{i} \eqdef \rmd \chi / \rmd \init(\epart{0}{i})$ for all $i$. 

Note that we may, in the light of Remark~\ref{we:vs:delmoral},  obtain particle approximations $(\epart{0:n}{i}, \ewght{n}{i})_{i = 1}^\N$, $n \in \nset$, of the smoothing distribution flow $(\post{0:n})_{n \in \nset}$ by applying the previous sampling scheme to the model formed by $\chi'$ and $(\uk{n}')_{n \in \nset}$. From an algorithmic point of view, it is easy to see that the only change needed is to insert, just after Line 3, the command $\epart{0:n + 1}{i} \gets (\epart{0:n}{\ind{n + 1}{i}}, \epart{n + 1}{i})$ storing the particle paths (in particular, the weight-updating step on Line~4 remains the same). Still, it is well known that repeated selection operations lead to coalescing paths $(\epart{0:n}{i})_{i = 1}^\N$, and as a consequence the variance of this naive smoothing estimator increases rapidly with $n$; indeed, in the case of additive state functionals, the growth in variance is typically quadratic in $n$ (see \cite{olsson:westerborn:2014b} for a discussion), which is unreasonable from a computational point of view. We will thus rely on more advanced, stochastically stable smoothing technology avoiding the particle-path degeneracy problem by taking advantage of the time-uniform convergence of the marginal samples $(\epart{n}{i})_{i = 1}^\N$. This will be discussed in the next section. 

Finally, we note that the re-weighting operation on Line~4 in Algorithm~\ref{alg:ideal:SMC} requires the transition density $\ud{n}$ to be tractable, which is not the case in general. We will return to the general case in Section~\ref{sec:pseudo:marginal:PaRIS}. 

%%%%%%%%
%% The PaRIS
%%%%%%%%

\subsubsection{Backward sampling} 
\label{sec:BS}

%The following quantities will play a key role in the following. 
For each $m \in \nset$, define the  \emph{backward Markov kernel} 
\begin{equation} \label{eq:def:backward:kernel}
    \bkw{m}(x_{m + 1}, \rmd x_m) \eqdef \frac{\post{m}(\rmd x_m) \, \ud{m}(x_m, x_{m + 1})}{\post{m}[\ud{m}(\cdot, x_{m + 1})]}
\end{equation}
on $\Xset_{m + 1} \times \Xfd_m$. In addition, for each $n \in \nsetpos$, let the Markov kernel   
\begin{equation} \label{eq:def:tstat}
\tstat{n}(x_n, \rmd x_{0:n - 1}) \eqdef \prod_{m = 0}^{n - 1} \bkw{m}(x_{m + 1}, \rmd x_m)
\end{equation}
on $\Xset_n \times \Xfd^{n - 1}$ denote the joint law of the backward Markov chain induced by the kernels \eqref{eq:def:backward:kernel} when initialised at $x_n \in \Xset_n$. An important class of sequential Monte Carlo joint-smoothing methods \cite{doucet2000sequential,godsill:doucet:west:2004} is based on the following result. 
\begin{lemma} 
\label{lem:reversibility}
\ 

\begin{itemize} 
\item[(i)]
For all $n \in \nset$ and $h \in \bmf{\Xfd_n \tensprod \Xfd_{n + 1}}$, 
\begin{equation} \label{eq:reversibility}
\iint \post{n}(\rmd x_n) \, \uk{n}(x_n, \rmd x_{n + 1}) \, h(x_n, x_{n + 1}) = \iint \post{n} \uk{n}(\rmd x_{n + 1}) \, \bkw{n}(x_{n + 1}, \rmd x_n) \, h(x_n, x_{n + 1}). 
\end{equation}
\item[(ii)]
For all $n \in \nsetpos$ and $h \in \bmf{\Xfd^n}$, 
$
\post{0:n} h = \post{n} \tstat{n} h. 
$
\end{itemize}
\end{lemma}

In the case of state-space models, the identity (ii) above is a well-known result typically referred to as the \emph{backward decomposition} of the joint-smoothing distribution; still, as far as known to the authors, it has never been established in the general setting considered in the present paper, and a proof of Lemma~\ref{lem:reversibility} is hence given in the supplement (Section~\ref{sec:proof:lem:reversibility}) for completeness. Importantly, as noted in \cite{cappe:2009}, the functions $(\tstat{n} h_n)_{n \in \nsetpos}$ can be expressed recursively through  
\begin{equation} \label{eq:forward:smoothing}
\tstat{n + 1} h_{n + 1}(x_{n + 1}) = \int (\tstat{n} h_n(x_n) + \addf{n}(x_n, x_{n + 1})) \, \bkw{n}(x_{n + 1}, \rmd x_n), \quad n \in \nset, 
\end{equation}
with, by convention, $\tstat{0} h_0 \equiv 0$. Here the backward kernel $\bkw{n}$ depends on the marginal $\post{n}$; thus, the recursion is driven by the marginal flow $(\post{n})_{n \in \nset}$, which may again be expressed recursively through \eqref{eq:recursion:FK:marg}. However, as these marginals are, as already mentioned, generally intractable, exact computations need typically to be replaced by approximations. The authors of \cite{delmoral:doucet:singh:2010} propose to approximate the values of each statistic $\tstat{n} h_n$ at a random, discrete support formed by particles. More precisely, assume again that the transition density $\ud{n}$ is tractable and, by induction, that at time step $n$ we have at hand a given particle sample $(\epart{n}{i}, \ewght{n}{i})_{i = 1}^\N$ and a set of statistics $(\tstat[i]{n})_{i = 1}^\N$ such that $\tstat[i]{n}$ is an approximation of $\tstat{n} h_n(\epart{n}{i})$. Then, in order to propagate the statistics $(\tstat[i]{n})_{i = 1}^\N$ forward, one updates, in a first substep, the particle sample $(\epart{n}{i}, \ewght{n}{i})_{i = 1}^\N$ recursively by forward sampling (Algorithm~\ref{alg:ideal:SMC}). After forward sampling, one replaces, in the definition \eqref{eq:def:backward:kernel} of $\bkw{n}$, $\post{n}$ by the corresponding particle approximation, yielding the updates 
\begin{equation} \label{eq:tstat:FFBSm:update}
\tstat[i]{n + 1} = \sum_{j = 1}^\N \trm{n}(i, j) (\tstat[j]{n} + \addf{n}(\epart{n}{j}, \epart{n + 1}{i})), \quad i \in \intvect{1}{\N}, 
\end{equation} 
where we have defined the transition kernel  
\begin{equation} \label{eq:def:trm}
\trm{n}(i, j) \eqdef \frac{\ewght{n}{j} \ud{n}(\epart{n}{j}, \epart{n + 1}{i})}{\sum_{j' = 1}^\N \ewght{n}{j'} \ud{n}(\epart{n}{j'}, \epart{n + 1}{i})}
\end{equation}
on $\intvect{1}{\N} \times \powerset{\intvect{1}{\N}}$. Since computing each $\tstat[i]{n}$ according to \eqref{eq:tstat:FFBSm:update} has a linear computational complexity in the number $\N$ of particles, the overall complexity this approach is \emph{quadratic} in $\N$. In order to deal with this significant computational burden, the authors of \cite{olsson:westerborn:2014b} suggest replacing summation by additional Monte Carlo simulation. More precisely, by sampling, for each $i$, $\K \in \nsetpos$ independent indices $(J^{(i, j)})_{j = 1}^\M$ from $\trm{n}(i, \cdot)$ and replacing \eqref{eq:tstat:FFBSm:update} by  
\begin{equation} \label{eq:PaRIS:update}
\tstat[i]{n + 1} = \frac{1}{\K} \sum_{j = 1}^{\K} \left( \tstat[\bi{n + 1}{i}{j}]{n} + \addf{n}(\epart{n}{\bi{n + 1}{i}{j}}, \epart{n + 1}{i}) \right), \quad i \in \intvect{1}{\N}, 
\end{equation}
the computational complexity can, as we shall soon see, be reduced significantly. At each iteration, the self-normalised estimator $\sumwght{n}^{-1} \sum_{i = 1}^\N \ewght{n}{i} \tstat[i]{n}$ serves as an estimator of the quantity $\post{n} \tstat{n} h_n = \post{0:n} h_n$ of interest. This second operation, which we will refer to as \emph{backward sampling}, 
$$
    (\tstat[i]{n + 1})_{i = 1}^\N \sim \mathsf{BS}((\epart{n}{i}, \tstat[i]{n}, \ewght{n}{i})_{i = 1}^\N, (\epart{n + 1}{i})_{i = 1}^\N),  
$$
is summarised in Algorithm~\ref{alg:ideal:BS}. 

\begin{algorithm}[h] 
    \KwData{$(\epart{n}{i}, \tstat[i]{n}, \ewght{n}{i})_{i = 1}^\N$, $(\epart{n + 1}{i})_{i = 1}^\N$}
    \KwResult{$(\tstat[i]{n + 1})_{i = 1}^\N$}
    \For{$i = 1 \to \N$}{
    \For{$j = 1 \to \K$}{
    draw $\bi{n + 1}{i}{j} \sim \trm{n}(i, \cdot)$\;
    }
    set $\tstat[i]{n + 1} \gets \frac{1}{\K} \sum_{j = 1}^{\K} \left( \tstat[\bi{n + 1}{i}{j}]{n} + \addf{n}(\epart{n}{\bi{n + 1}{i}{j}}, \epart{n + 1}{i}) \right)$\;
}
\caption{Backward sampling, \textsf{BS}} \label{alg:ideal:BS}
\end{algorithm}

Let us examine the sampling step on Line~3 in Algorithm~\ref{alg:ideal:BS} more closely. 
In order to keep the algorithmic complexity at a reasonable level, the computation of the normalising constant of $\trm{n}(i, \cdot)$, which consists of $\N$ terms, should be avoided; otherwise, the overall complexity remains quadratic in $\N$. This is possible using, \eg,   
\begin{itemize}
\item[--] \emph{rejection sampling}. This approach relies on the mild assumption that there exists some measurable function $c$ on $\Xset_{n + 1}$ such that $\ud{n}(x_n, x_{n + 1}) \leq c(x_{n + 1})$ for all $x_{n: n + 1} \in \Xset_n \times \Xset_{n + 1}$. Then, following \cite{douc:garivier:moulines:olsson:2010}, $\trm{n}(i, \cdot)$ can be sampled from by generating a candidate $J^\ast$ from $\cat(\{ \ewght{n}{\ell}\}_{\ell = 1}^\N)$ and accepting the same with probability 
\begin{equation} \label{eq:std:acc:prob:backward:sampling}
\accprob \eqdef \frac{\ud{n}(\epart{n}{J^\ast}, \epart{n + 1}{i})}{c(\epart{n + 1}{i})}. 
\end{equation}
The procedure is repeated until acceptance, conditionally on which $J^\ast$ is distributed according to $\trm{n}(i, \cdot)$. Since the $\cat(\{ \ewght{n}{\ell}\}_{\ell = 1}^\N)$ distribution is independent of $i$, this circumvents the need to compute a normalising sum for every $i$. The approach may significantly reduce the computational complexity; indeed, as shown in \cite[Proposition~2]{douc:garivier:moulines:olsson:2010}, the expected overall complexity of the algorithm is \emph{linear} in $\N$ under certain assumptions.   
\item[--] \emph{MCMC methods}. Another possibility is to generate the variables $(\bi{n + 1}{i}{j})_{j = 1}^{\K}$ using the Metropolis-Hastings algorithm. For this purpose, let $\rho$ be some proposal transition density on $\intvect{1}{\N}^2$. Then proceeding recursively, given $\bi{n + 1}{i}{j} = J$, a candidate $J^\ast$ for $\bi{n + 1}{i}{j + 1}$ is sampled from the density $\rho(J, \cdot)$ and accepted with probability 
$$
\accprob[MH] \eqdef 1 \wedge\frac{\ewght{n}{J^\ast} \ud{n}(\epart{n}{J^\ast}, \epart{n + 1}{i}) \rho(J^\ast, J)}{\ewght{n}{J} \ud{n}(\epart{n}{J}, \epart{n + 1}{i}) \rho(J, J^\ast)}. 
$$ 
If rejection, then we set $\bi{n + 1}{i}{j + 1} = J$. An appealing solution is to let $\rho$ take the form of an independent proposal given by the $\cat(\{ \ewght{n}{i} \}_{i = 1}^\N)$ distribution; in that case $\accprob[MH]$ simplifies to  
\begin{equation} \label{eq:std:MH:prob:backward:sampling}
\accprob[MH] = 1 \wedge \frac{\ud{n}(\epart{n}{J^\ast}, \epart{n + 1}{i})}{\ud{n}(\epart{n}{J}, \epart{n + 1}{i})}. 
\end{equation}
With this approach, the variables $(\bi{n + 1}{i}{j})_{j = 1}^{\K}$ are conditionally dependent; this can be counteracted by including only an $m$-skeleton of this sequence in the update \eqref{eq:PaRIS:update}. An important advantage of this approach over rejection sampling is that it does not require $\ud{n}$ to be dominated.  
\end{itemize}

Finally, combining the forward and backward sampling operations in accordance with Algorithm~\ref{alg:ideal:PaRIS} yields a generalisation of the PaRIS proposed in \cite{olsson:westerborn:2014b} to a general framework comprising Feynman-Kac models and auxiliary particle filters. 

\begin{algorithm}[h] 
    \KwData{$(\epart{n}{i}, \tstat[i]{n}, \ewght{n}{i})_{i = 1}^\N$}
    \KwResult{$(\epart{n + 1}{i}, \tstat[i]{n + 1}, \ewght{n + 1}{i})_{i = 1}^\N$}
    run $(\epart{n + 1}{i}, \ewght{n + 1}{i})_{i = 1}^\N \sim \mathsf{FS}((\epart{n}{i}, \ewght{n}{i})_{i = 1}^\N)$\;
    run $(\tstat[i]{n + 1})_{i = 1}^\N \sim \mathsf{BS}((\epart{n}{i}, \tstat[i]{n}, \ewght{n}{i})_{i = 1}^\N, (\epart{n + 1}{i})_{i = 1}^\N)$\;
    \medskip
\caption{Full PaRIS update.} \label{alg:ideal:PaRIS}
\end{algorithm}

Algorithm~\ref{alg:ideal:PaRIS} is initialised by drawing $(\epart{0}{i})_{i = 1}^\N \sim \init^{\varotimes \N}$ and letting $\ewght{0}{i} = \rmd \chi / \rmd \init(\epart{0}{i})$ and $\tstat[i]{n} = 0$.  

In this scheme, the sample size $\K$ of the backward sampling operation is an algorithmic parameter that has to be set by the user. As shown in Section~\ref{sec:theoretical:results}, the produced estimators are, for all $n \in \nset$, consistent and asymptotically normal for any fixed $\K$ larger than or equal to one. In addition, for any fixed $\K \geq 2$ the algorithm is stochastically stable with an $\mathcal{O}(n)$ variance, which is optimal; see \cite[Section~1]{olsson:westerborn:2014b} for a discussion. These results form a nontrivial extension of similar results obtained by \cite{olsson:westerborn:2014b} in the simpler setting of state-space models and bootstrap particle filters. 

Finally, we remind the reader that we have here considered the idealised situation where the unnormalised transition densities $(\ud{n})_{n \in \nset}$ can be evaluated pointwise, which will generally not be the case for the applications we will consider. Thus, in the next section we will approach the more general case where these transition densities are intractable but may be estimated, and we will show how consistent, asymptotically normal, and stochastically stable estimators can be produced also in such a scenario by pseudo-marginalising the forward and backward sampling operations separately. 
 


\section{Pseudo-marginal PaRIS algorithms}
\label{sec:pseudo:marginal:PaRIS}
\subsection{Pseudo marginalisation in Monte Carlo methods}
\label{sec:pseudo:marginalisation}

Pseudo-marginalisation was originally proposed in \cite{beaumont:2003} in the framework of MCMC methods, and in \cite{andrieu:robert:2009} the method was developed further and provided with a solid theoretical basis. In the following we recapitulate briefly the main idea behind this approach. Consider the problem of sampling from some target distribution $\pi$ defined on some measurable space $(\Xset, \Xfd)$ and having a density with respect to some reference measure $\mu$. This density is assumed to be proportional to some \emph{intractable} nonnegative measurable function $\ell$ on $\Xset$, \ie, $\pi(\rmd x) = \lambda(\rmd x) / \lambda \1_\Xset$, where $\lambda(\rmd x) \eqdef \ell(x) \, \mu(\rmd x)$ is finite. While the target density is intractable we assume that there exist some additional state space $(\Zset, \Zfd)$, a Markov kernel $\mathbf{R}$ on $\Xset \times \Zfd$, and some nonnegative measurable function $\Xset \times \Zset \ni (x, z) \mapsto \ell \langle z \rangle (x)$ known up to a constant of proportionality and such that for all $x \in \Xset$, 
\begin{equation} \label{eq:pseudo:marginalisation}
\int \ell \langle z \rangle(x) \, \mathbf{R}(x, \rmd z) = \ell(x). 
\end{equation}
Thus, a pointwise estimate of $\ell(x)$ can be obtained by generating $\zeta$ from $\mathbf{R}(x, \rmd z)$ and computing the statistic $\ell \langle \zeta \rangle(x)$, the \emph{pseudo marginal}. In Monte Carlo methods, the operation of replacing, when necessary, the true marginal $\ell$ by its pseudo marginal is referred to as \emph{pseudo marginalisation}. Interestingly, even though pseudo marginalisation is based on the plug-in principle, it preserves typically the consistency of an algorithm. In order to see the this, let $\bar{\mathsf{X}} \eqdef \mathsf{X} \times \mathsf{Z}$ and $\bar{\mathcal{X}} \eqdef \mathcal{X} \tensprod \mathcal{Z}$; then one may define an extended target distribution $\bar{\pi}(\rmd \bar{x}) \eqdef \bar{\lambda}(\rmd \bar{x}) / \bar{\lambda} \1_{\bar{\mathsf{X}}} = \bar{\lambda}(\rmd \bar{x}) / \lambda \1_{\mathsf{X}}$ on $(\bar{\mathsf{X}}, \bar{\mathcal{X}})$, where 
\begin{equation} \label{eq:extended:target}
\bar{\lambda}(\rmd \bar{x}) \eqdef \ell \langle z \rangle(x) \, \kernel{R}(x, \rmd z) \, \mu(\rmd x)
\end{equation}
(with $\bar{x} = (x, z)$). By \eqref{eq:pseudo:marginalisation}, $\pi$ is the marginal of $\bar{\pi}$ with respect to the $x$ component. This means that we may produce a random sample $(\xi^i)_{i = 1}^N$ in $\Xset$ targeting $\pi$ by generating a sample $(\xi^i, \zeta^i)_{i = 1}^N$ targeting $\bar{\pi}$ and simply discarding the $\Zset$-valued variables $(\zeta^i)_{i = 1}^N$. Let $\rho$ be a Markov transition density on $\Xset$ with respect to the reference measure $\mu$. Then following \cite{andrieu:robert:2009}, a Markov chain $(\xi_m, \zeta_m)_{m \in \nset}$ targeting $\bar{\pi}$ can be produced on the basis of the Metropolis-Hastings algorithm as follows. Given a state $(\xi_m, \zeta_m)$, a candidate $(\xi^\ast, \zeta^\ast)$ for the next state is generated by drawing $\xi^\ast \sim \rho(x) \, \mu(\rmd x)$ and $\zeta^\ast \sim \kernel{R}(\xi^\ast, \rmd z)$ and accepting the same with probability 
$$
\alpha \eqdef 1 \wedge \frac{\ell \langle \zeta^\ast \rangle(\xi^\ast) \rho(\xi^\ast, \xi_m)}{\ell \langle \zeta_m \rangle (\xi_m) \rho(\xi_m, \xi^\ast)},  
$$
which is tractable. Note that $\alpha$ is indeed a pseudo-marginal version of the exact acceptance probability $1 \wedge \ell (\xi^\ast) \rho(\xi^\ast, \xi_m) / \ell (\xi_m) \rho(\xi_m, \xi^\ast)$ obtained if $\ell$ was known. Note that the auxiliary variable $\zeta$ enters $\alpha$ only through the estimates $\ell \langle \zeta^\ast \rangle (\xi^\ast)$ and $\ell \langle \zeta_m \rangle (\xi_m)$, and since the latter has already been computed at the previous iteration there is no need of recomputing this quantity. In the \emph{Monte-Carlo-within-Metropolis algorithm} (see again \cite{andrieu:robert:2009}) $\ell \langle z \rangle(x)$ is a pointwise importance sampling estimate of $\ell(x)$ based on a Monte Carlo sample $\zeta$ generated from $\mathbf{R}(x, \cdot)$. Alternatively, the extended distribution $\bar{\pi}$ can be sampled using rejection sampling or importance sampling, leading to consistent pseudo-marginal formulations of these algorithms as well. 

In the present paper we will generalise the pseudo-marginal approach towards biased estimation by allowing the function 
\begin{equation} \label{eq:biased:pseudo:marginalisation}
    \ell^\precpar(x) \eqdef \int \ell \langle z \rangle(x) \, \mathbf{R}(x, \rmd z)
\end{equation}
on $\Xset$ to be different from $\ell$. Here $\varepsilon \geq 0$ is some accuracy parameter describing the distance between $\ell^\precpar$ and $\ell$. Such biased estimates appear naturally, e.g. when the law $\pi$ is governed by a diffusion process and the density $\ell$ is approximated on the basis of different discretisation schemes; see the next section. In that case, $\precpar$ plays the role of the discretisation step size. In \eqref{eq:biased:pseudo:marginalisation}, also the estimator $\ell \langle z \rangle(x)$ and the kernel $\mathbf{R}(x, \rmd z)$ may depend on $\precpar$, even though this is suppressed in the notation. By introducing the possibly unnormalised measure $\lambda^\varepsilon(\rmd x) \eqdef \ell^\precpar(\rmd x) \, \mu(\rmd x)$ on $\Xfd$ we may define the \emph{skew} target probability measure
$$
    \pi^\varepsilon(\rmd x) \eqdef \frac{\lambda^\precpar(\rmd x)}{\lambda^\precpar \1_\Xset}
$$
on $\Xfd$. In the biased case, generating a sample $(\xi^i, \zeta^i, \omega^i)_{i = 1}^N$ targeting \eqref{eq:extended:target} will, by \eqref{eq:biased:pseudo:marginalisation}, provide a sample $(\xi^i, \omega^i)_{i = 1}^N$ targeting $\pi^\varepsilon$ as a by-product. Thus, it is of utmost importance to obtain control over the bias between $\pi$ and $\pi^\precpar$, which is possible under the assumption that there exists a constant $c \geq 0$ such that for all $h \in \bmf{\Xfd}$ and $\varepsilon$,   
\begin{equation} \label{eq:lipschitz:simple:case}
    \left| \lambda^\varepsilon h - \lambda h \right| \leq c \varepsilon \| h \|_\infty. 
\end{equation}
For instance, in the diffusion process case mentioned above, a condition of type \eqref{eq:lipschitz:simple:case} holds, as we will see in Section~\ref{sec:theoretical:results}, typically true in the case where the density is approximated using the \emph{Durham-Gallant estimator} \cite{durham:gallant:2002}. Using that for all $h \in \bmf{\Xfd}$,  
$$
    \pi^\precpar h - \pi h = \pi^\precpar h \left( 1 - \frac{\lambda^\varepsilon \1_\Xset}{\lambda \1_\Xset} \right) + \frac{\lambda^\precpar h - \lambda h}{\lambda \1_\Xset}, 
$$
we straightforwardly obtain the bound  
$$
    \left| \pi^\precpar h - \pi h \right| \leq \precpar \frac{2 c}{\lambda \1_\Xset} \| h \|_\infty
$$
on the systematic error induced by the skew model. Note that the unbiased case \eqref{eq:pseudo:marginalisation} corresponds to letting $\precpar = 0$ in assumption~\eqref{eq:lipschitz:simple:case}. In the next section we will present a solution to the main problem addressed in Section~\ref{sec:model} exploring a pseudo-marginalised version of the PaRIS discussed in Section~\ref{sec:PaRIS}. 

\subsection{Pseudo-marginal PaRIS}

The algorithm that we will propose relies on the following assumption. 
 
\begin{hypH}
\label{assum:biased:estimate}
Let $(\Zset_n, \Zfd_n)_{n \in \nsetpos}$ be a sequence of general state spaces. For each $n \in \nset$ there exist a Markov kernel $\ukdist{n}$ on $\Xset_n \times \Xset_{n + 1} \times \Zfd_{n + 1}$ and a positive measurable function $\ukest{n}{z}(x_n, x_{n + 1})$ on $\Xset_n \times \Xset_{n + 1} \times \Zset_{n + 1}$ such that for every $x_{n:n + 1} \in \Xset_n \times \Xset_{n + 1}$, drawing $\zeta \sim \ukdist{n}(x_{n:n + 1}, \rmd z)$ and computing $\ukest{n}{\zeta}(x_n, x_{n + 1})$ yields an estimate of $\ud{n}(x_n, x_{n + 1})$.  
\end{hypH}


Under \hypref{assum:biased:estimate}, we denote, for every $n \in \nset$ and $x_{n:n + 1} \in \Xset_n \times \Xset_{n + 1}$, by   
\begin{equation} \label{eq:def:udmod}
\udmod{n}(x_n, x_{n + 1}) \eqdef \int \ukdist{n}(x_{n:n + 1}, \rmd z) \, \ukest{n}{z}(x_n, x_{n + 1})
\end{equation}
the expectation of the statistic $\ukest{n}{\zeta}(x_n, x_{n + 1})$. Here $\precpar$ is an accuracy parameter belonging to some parameter space $\precparsp \subset \rset$ and controlling the bias of the estimated model with respect to the true model; this will be discussed in depth in Section~\ref{sec:lipschitz:continuity}. For each $n \in \nset$ we define the unnormalised kernel 
\begin{equation} \label{eq:def:ukmod}
    \ukmod{n}(x_n, \rmd x_{n + 1}) = \udmod{n}(x_n, x_{n + 1}) \, \mu_{n + 1}(\rmd x_{n + 1}) 
\end{equation}
on $\Xset_n \times \Xfd_{n + 1}$. 
 
%%%%%%
%% pmFS
%%%%%%

\subsubsection{Pseudo-marginal forward sampling}
\label{eq:sec:pm:forward:sampling}

In the case where each $\ud{n}$ is intractable, so is the mixture distribution $\partmixt_n$ defined in \eqref{eq:def:partmixt}. Under \hypref{assum:biased:estimate}, we may, as in Section~\ref{sec:pseudo:marginalisation}, aim at consistent pseudo-marginalisation of the forward-sampling operation by applying self-normalised importance sampling to the extended mixture  
$$
\bar{\pi}_n(i, \rmd x, \rmd z)
\propto \ewght{n}{i} \ukest{n}{z}(\epart{n}{i}, x) \, \mu_{n + 1}(\rmd x) \, \ukdist{n}(\epart{n}{i}, x, \rmd z) 
$$
on $\bar{\Xfd}_{n + 1} \eqdef \powerset{\intvect{1}{\N}} \tensprod \Xfd_{n + 1} \tensprod \Zfd_{n + 1}$ using the instrumental distribution  
$$
\bar{\rho}_n(i, \rmd x, \rmd z) \propto \ewght{n}{i} \adjfuncforward{n}(\epart{n}{i}) \, \prop{n}(\epart{n}{i}, \rmd x) \, \ukdist{n}(\epart{n}{i}, x, \rmd z)
$$
on the same space. Note that the marginal of $\bar{\pi}_n$ with respect to $(i, x)$ is the distribution proportional to $\ewght{n}{i} \ukmod{n}(\epart{n}{i}, \rmd x)$, whose distance to the target $\pi_n$ of interest is controlled by the precision parameter $\precpar$. Here the adjustment multiplier $\adjfuncforward{n}$ and the proposal kernel $\prop{n}$ of the instrumental distribution are as in Section~\ref{sec:SMC}. Each draw from $\bar{\rho}_n$ is assigned an importance weight given by the (tractable) Radon--Nikodym derivative of $\bar{\pi}_n$ with respect to $\bar{\rho}_n$. It is easily seen that this sampling operation, which is detailed in Algorithm~\ref{alg:pm:SMC}, corresponds to replacing the intractable transition density $\ud{n}$ on Line~4 in Algorithm~\ref{alg:ideal:SMC} by an estimate provided by \hypref{assum:biased:estimate}; we will hence refer to Algorithm~\ref{alg:pm:SMC} as \emph{pseudo-marginal forward sampling} and express it concisely as 
$$
    (\epart{n + 1}{i}, \ewght{n + 1}{i})_{i = 1}^\N \sim \mathsf{pmFS}((\epart{n}{i}, \ewght{n}{i})_{i = 1}^\N). 
$$

\begin{algorithm}[h] 
    \KwData{$(\epart{n}{i}, \ewght{n}{i})_{i = 1}^\N$}
    \KwResult{$(\epart{n + 1}{i}, \ewght{n + 1}{i})_{i = 1}^\N$}
    \For {$i = 1 \to \N$}{
        draw $\ind{n + 1}{i} \sim \cat(\{ \adjfuncforward{n}(\epart{n}{\ell}) \ewght{n}{\ell} \}_{\ell = 1}^\N)$\;
        draw $\epart{n + 1}{i} \sim \prop{n}(\epart{n}{{\ind{n + 1}{i}}}, \cdot)$\;
        draw $\zpart{n + 1}{i} \sim \ukdist{n}(\epart{n}{{\ind{n + 1}{i}}}, \epart{n + 1}{i}, \cdot)$\;
        set $\displaystyle 
        \ewght{n + 1}{i} \gets \frac{\ukest{n}{\zpart{n + 1}{i}}(\epart{n}{{\ind{n + 1}{i}}}, \epart{n + 1}{i})}{\adjfuncforward{n}(\epart{n}{{\ind{n + 1}{i}}}) \propdens{n}(\epart{n}{{\ind{n + 1}{i}}}, \epart{n + 1}{i})}$\;
}
\caption{Pseudo-marginal forward sampling, \textsf{pmFS}.} \label{alg:pm:SMC}
\end{algorithm}

Iterating recursively, after initialisation as in Section~\ref{sec:SMC}, pseudo-marginal forward sampling yields a generalisation of the random-weight particle filter proposed in \cite{fearnhead2008particle} in the context of partially observed diffusion processes. 

\subsubsection{Pseudo-marginal backward sampling}
\label{eq:sec:backward:sampling:pseudo:marg}

Let us turn our focus to backward sampling. As intractability of $\ud{n}$ implies intractability of the kernel $\trm{n}$ (defined in \eqref{eq:def:trm}), we aim at further pseudo marginalisation by embedding $\trm{n}$ into the extended kernel  
$$
\trmext{n}(i, j, \rmd z) \propto \ewght{n}{j} \ukest{n}{z}(\epart{n}{j}, \epart{n + 1}{i}) \, \ukdist{n}(\epart{n}{j}, \epart{n + 1}{i}, \rmd z)
$$
on $\intvect{1}{\N} \times \powerset{\intvect{1}{\N}} \tensprod \Zfd_{n + 1}$. For every $i$, the marginal of $\trmext{n}(i, \cdot)$ with respect to the $j$ component is, by \eqref{eq:def:udmod}, proportional to $\ewght{n}{j} \udmod{n}(\epart{n}{j}, \epart{n + 1}{i})$, a distribution that we expect to be close to $\trm{n}(i, \cdot)$ for small $\precpar$. The intractable sampling step on Line~3 in Algorithm~\ref{alg:ideal:BS} can therefore be replaced by sampling from $\trmext{n}(i, \cdot)$, after which the auxiliary variables are discarded. The latter sampling operation will be examined in detail in the next section. This approach, which we express concisely as 
$$
    (\tstat[i]{n + 1})_{i = 1}^\N \sim \mathsf{pmBS}((\epart{n}{i}, \tstat[i]{n}, \ewght{n}{i})_{i = 1}^\N, (\epart{n + 1}{i})_{i = 1}^\N), 
$$
is summarised in Algorithm~\ref{alg:pm:backward:sampling}.  

\begin{algorithm}[h] 
    \KwData{$(\epart{n}{i}, \tstat[i]{n}, \ewght{n}{i})_{i = 1}^\N$, $(\epart{n + 1}{i})_{i = 1}^\N$}
    \KwResult{$(\tstat[i]{n + 1})_{i = 1}^\N$}
    \For{$i = 1 \to \N$}{
    \For{$j = 1 \to \K$}{
    draw $( \bi{n + 1}{i}{j}, \zpart{n + 1}{(i, j)}) \sim \trmext{n}(i, \cdot)$\;
    }
    set $\tstat[i]{n + 1} \gets \frac{1}{\K} \sum_{j = 1}^{\K} \left( \tstat[\bi{n + 1}{i}{j}]{n} + \addf{n}(\epart{n}{\bi{n + 1}{i}{j}}, \epart{n + 1}{i}) \right)$\;
}
\caption{Pseudo-marginal backward sampling, \textsf{pmBS}.} \label{alg:pm:backward:sampling}
\end{algorithm}

It remains to discuss how to sample from the extended distribution $\trmext{n}(i, \cdot)$. 
In the following we propose two possible approaches, which can be viewed as pseudo-marginal versions of the techniques discussed in Section~\ref{sec:BS}.  

\subsubsection*{Rejection sampling from $\trmext{n}$} 

Assume that there exists some measurable nonnegative function $c$ on $\Xset_{n + 1}$ such that for all $(x_{n:n + 1}, z) \in \Xset_n \times \Xset_{n + 1} \times \Zset_{n + 1}$,
$
\ukest{n}{z}(x_n, x_{n + 1}) \leq c(x_{n + 1}).  
$
Since this condition allows the Radon--Nikodym derivative of $\trmext{n}(i, \cdot)$ with respect to the probability measure 
\begin{equation} \label{eq:proposal:pm:rejection}
\rho_n^i(j, \rmd z) \propto \ewght{n}{j} \kernel{R}(\epart{n}{j}, \epart{n + 1}{i}, \rmd z)
\end{equation} 
on $\powerset{\intvect{1}{\N}} \tensprod \Zfd_{n + 1}$ to be bounded uniformly as 
$$
\frac{\rmd \trmext{n}(i, \cdot)}{\rmd \rho_n^i}(j, z) \leq \frac{c(\epart{n + 1}{i}) \sumwght{n}}{\sum_{i' = 1}^\N \ewght{n}{i'} \udmod{n}(\epart{n}{i'}, \epart{n + 1}{i})},  
$$ 
we may sample from the target $\trmext{n}(i, \cdot)$ using rejection sampling. Thus, the following procedure is iterated until acceptance: simulate a candidate $(J^\ast, \zeta^\ast)$ from $\rho_n^i$ by drawing $J^\ast \sim \cat(\{ \ewght{n}{\ell} \}_{\ell = 1}^\N)$ and $\zeta^\ast \sim \kernel{R}(\epart{n}{J^\ast}, \epart{n + 1}{i}, \rmd z)$; then accept the same with (tractable) probability 
$$
\accprobext \eqdef \frac{\ukest{n}{\zeta^\ast}(\epart{n}{J^\ast}, \epart{n + 1}{i})}{c(\epart{n + 1}{i})}. 
$$ 
Then conditionally to acceptance, the candidate has distribution $\trmext{n}(i, \cdot)$.  
Notably, the probability $\accprobext$ is obtained by simply plugging a transition density estimate provided by \hypref{assum:biased:estimate} into the probability $\accprob$ (see \eqref{eq:std:acc:prob:backward:sampling}) corresponding to the case where $\ud{n}$ is known. Moreover, since the proposal density is independent of $i$, the expected complexity of this sampling schedule is linear in the number of particles (we refer again to \cite{douc:garivier:moulines:olsson:2010}).  

\subsubsection*{MCMC sampling from $\trmext{n}$}

In some cases, bounding the estimator $\ukest{n}{z}(x_n, x_{n + 1})$ uniformly in $z$ and $x_n$ is not possible. Still, we may sample from $\trmext{n}(i, \cdot)$ using the Metropolis-Hastings algorithm with $\rho_n$ (in \eqref{eq:proposal:pm:rejection}) as independent proposal. In this case, $(\bi{n + 1}{i}{j}, \zpart{n + 1}{(i, j)})_{j = 1}^{\K}$ is a Markov chain generated recursively by the following mechanism. Given a state $\bi{n + 1}{i}{j} = J$ and $\zpart{n + 1}{(i, j)} = \zeta$, a candidate $(J^\ast, \zeta^\ast)$ for the next state is drawn from $\rho_n$ (as described above) and accepted with probability 
$$
\accprobext[MH] \eqdef 1 \wedge \frac{\ukest{n}{\zeta^\ast}(\epart{n}{J^\ast}, \epart{n + 1}{i})}{\ukest{n}{\zeta}(\epart{n}{J}, \epart{n + 1}{i})}. 
$$  
In the case of rejection, the next state is assigned the previous state. The resulting Markov chain has $\trmext{n}(i, \cdot)$ as stationary distribution and similar to the case of rejection sampling, the acceptance probability $\accprobext[MH]$ can be viewed as a plug-in estimate of the corresponding probability $\accprob[MH]$ (see \eqref{eq:std:MH:prob:backward:sampling}). 

\subsubsection{Pseudo-marginal PaRIS: full update} 

Combining the pseudo-marginal forward and backward sampling operations yields a pseudo-marginal PaRIS update described in the following algorithm, which is the main contribution of this section.  

\begin{algorithm}[h] 
    \KwData{$(\epart{n}{i}, \tstat[i]{n}, \ewght{n}{i})_{i = 1}^\N$}
    \KwResult{$(\epart{n + 1}{i}, \tstat[i]{n + 1}, \ewght{n + 1}{i})_{i = 1}^\N$}
    run $(\epart{n + 1}{i}, \ewght{n + 1}{i})_{i = 1}^\N \sim \mathsf{pmFS}((\epart{n}{i}, \ewght{n}{i})_{i = 1}^\N)$\;
    run $(\tstat[i]{n + 1})_{i = 1}^\N \sim \mathsf{pmBS}((\epart{n}{i}, \tstat[i]{n}, \ewght{n}{i})_{i = 1}^\N, (\epart{n + 1}{i})_{i = 1}^\N)$\; 
    \medskip
\caption{Full pseudo-marginal PaRIS update.} \label{alg:pm:PaRIS}
\end{algorithm}

Algorithm~\ref{alg:pm:PaRIS} is initialised by drawing $(\epart{0}{i})_{i = 1}^\N \sim \chi^{\varotimes \N}$ and letting $\ewght{0}{i} = \rmd \chi / \rmd \init(\epart{0}{i})$ and $\tstat[i]{n} = 0$.

We now illustrate \hypref{assum:biased:estimate} by a few examples. 

\begin{example}[Durham--Gallant estimator]
\label{eq:durham:gallant}
As an illustration, we return to the state-space model framework discussed in Example~\ref{ex:state-space:models}. Let $\Xset \eqdef \rset^{d_x}$ and $\Yset \eqdef \rset^{d_y}$ be equipped with their respective Borel $\sigma$-fields $\Xfd$ and $\Yfd$, and let $(X_t)_{t > 0}$ be some diffusion process on $\Xset$ driven by the homogeneous stochastic differential equation
\begin{equation} \label{eq:SDE}
\rmd X_t = \mu(X_t) \, \rmd t + \sigma(X_t) \, \rmd W_t, \quad t > 0, 
\end{equation}
where $X_0 = 0$, $(W_t)_{t > 0}$ is $d_x$-dimensional Brownian motion, $b : \Xset \to \Xset$ and $\sigma : \Xset \to \Xset^2$ are twice differentiable with bounded first and second order derivatives. In addition, the matrix $\sigma \sigma^\intercal$ is uniformly non-degenerate. Let $(\mathcal{F}_t)_{t > 0}$ be the natural filtration generated by the process $(X_t)_{t > 0}$. The state sequence $(X_t)_{t > 0}$ is latent but partially observed at discrete time points $(t_n)_{n \in \nsetpos}$ which are assumed to be equally spaced for simplicity, \ie, $t_n = t_1 + \delta (n - 1)$ for all $n$ and some $\delta > 0$. Abusing notations, we denote $X_n \eqdef X_{t_n}$ and let $q_\delta$ be the transition density of $(X_n)_{n \in \nsetpos}$. Denote by $\kernel{Q}_\delta$ the transition kernel induced by $q_\delta$. In general, $q_\delta$ is intractable, which makes the problem of computing online, for a given data stream $(y_n)_{n \in \nsetpos}$ in $\Yset$, the sequence of joint-smoothing distributions \eqref{eq:smooth} in models of this sort very challenging. Still, using the Euler scheme, one may, for small $\delta$, approximate $q_\delta$ by 
$$
\bar{q}_\delta(x_n, x_{n + 1}) \eqdef \phi(x_{n + 1}; x_n + \delta \mu(x_n), \delta \sigma^2(x_n)), 
$$ 
where $\phi(\cdot; m, s^2)$ is the density of the Gaussian distribution with mean $m$ and variance $s^2$. Let $\bar{\kernel{Q}}_\delta$ be the transition kernel induced by $\bar{q}_\delta$. Since the approximation $\bar{q}_\delta$ is poor for $\delta$ not small enough, we may instead, as suggested in \cite{durham:gallant:2002}, pick some finer step size $\precpar \in \precparsp_\delta \eqdef \{ \delta / n : n \in \nsetpos \}$ and estimate the density $q_\delta(x_n, x_{n + 1})$ by $q_\delta \langle \zeta \rangle (x_n, x_{n + 1})$, where $\zeta = (\zeta^i)_{i = 1}^L$ are independent draws from some proposal $r(x_n, x_{n + 1}, z) \, \rmd z$ on $\Xset^2 \times \Xfd^{\delta / \precpar - 1}$ and  
$$
q_\delta \langle z \rangle (x_n, x_{n + 1}) \eqdef \frac{1}{L} \sum_{i = 1}^L \frac{\prod_{k = 1}^{\delta / \precpar} \bar{q}_\precpar(z_{k - 1}^i, z_k^i)}{r(x_n, x_{n + 1}, z^i)},  
$$
with $z^i = (z_1^i, \ldots, z_{\delta / \precpar - 1}^i)$ and, by convention, $z_0^i = x_n$ and $z_{\delta / \precpar}^i = x_{n + 1}$. In \cite{durham:gallant:2002}, $r(x_n, x_{n + 1}, z) \, \rmd z$ is the distribution of a discretised (possibly modified) \emph{Brownian bridge}, \ie, Brownian motion started at $x_n$ and conditioned to terminate at $x_{n + 1}$. 

Let $Y_n$ denote the $\Yset$-valued observation at time $t_n$. We will consider two different models for the observation process $(Y_n)_{n \in \nset}$.    

\subsubsection*{Case~1}
First, assume that for all $n \in \nsetpos$, 
$$
Y_n \mid \mathcal{F}_{t_n} \sim \md{n - 1}(X_{n - 1}, X_n, y_n) \, \rmd y_n, 
$$
where $\md{n - 1}$ is some tractable transition density with respect to Lebesgue measure. In this case, \hypref{assum:biased:estimate} holds with the estimator $\ukest{n}{z}(x_n, x_{n + 1}) = q_\delta \langle z \rangle (x_n, x_{n + 1}) \md{n}(x_n, x_{n + 1}, y_{n + 1})$ and the instrumental kernel $\ukdist{n}(x_n, x_{n + 1}, \rmd z) = \prod_{i = 1}^L r(x_n, x_{n + 1}, z^i) \, \rmd z^i$. Finally, we note that 
\begin{equation} \label{eq:ukmod:durham:gallant}
\ukmod{n}(x_n, \rmd x_{n + 1}) = \bar{\kernel{Q}}_\precpar^{\delta / \precpar}(x_n, \rmd x_{n + 1}) \, \md{n}(x_n, x_{n + 1}, y_{n + 1}), 
\end{equation}
which is generally intractable.  

\subsubsection*{Case 2} Alternatively, we may assume that $(Y_n)_{n \in \nsetpos}$ are discrete observations of the solution to the stochastic differential equation 
$$
\rmd Y_t = \tilde{\mu}(X_t, Y_t) \, \rmd t + \tilde{\sigma}(X_t, Y_t) \, \rmd \tilde{W}_t, \quad t > 0, 
$$
where $Y = 0$, $(\tilde{W}_t)_{t > 0}$ is $d_y$-dimensional Brownian motion independent of $(W_t)_{t > 0}$ and $\tilde{\mu} : \Xset \times \Yset \to \Yset$ and $\tilde{\sigma} : \Xset \times \Yset \to \Yset^2$ are known functions which are twice differentiable with bounded first and second order derivatives. In addition, the matrix $\tilde{\sigma} \tilde{\sigma}^\intercal$ is uniformly non-degenerate. Denote by $p_\delta$ the transition density of $(X_t, Y_t)_{t > 0}$. In this case, the joint-smoothing distributions can, for a given data stream $(y_n)_{n \in \nsetpos}$, be expressed as path measures \eqref{eq:FK:path} induced by $\ud{n}(x_n, x_{n + 1}) = p_\delta(x_n, y_n, x_{n + 1}, y_{n + 1})$, $n \in \nset$. Since also $p_\delta$ is generally intractable we subject the bivariate process to Euler discretisation, yielding the approximation 
\begin{multline}
\bar{p}_\delta(x_n, y_n, x_{n + 1}, y_{n + 1}) \\
\eqdef \phi \left( x_{n + 1}, y_{n + 1}; (x_n + \delta \mu(x_n), y_n + \delta \tilde{\mu}(x_n, y_n))^\intercal, \delta \, \mbox{diag}(\sigma^2(x_n), \tilde{\sigma}^2(x_n, y_n)) \right) \label{eq:td:bivariate:process}
\end{multline}
of $p_\delta$. Denote by $\bar{\kernel{P}}_\delta$ the Markov kernel induced by $\bar{p}_\delta$. In the case of sparse observations we may improve the approximation $\bar{p}_\delta$ by picking again some finer step size $\precpar \in \mathcal{E}_\delta$ and computing (by swapping, in~\eqref{eq:ukmod:durham:gallant}, $\bar{q}_\delta$ for $\bar{p}_\delta$ and letting $r(x_n, y_n, x_{n + 1}, y_{n + 1}, z) \, \rmd z$ be the distribution of a discretised, $\rset^{d_x + d_y}$-valued Brownian bridge started in $(x_n, y_n)$ and conditioned to terminate in $(x_{n + 1}, y_{n + 1})$) the Durham--Gallant estimator $p_\delta \langle z \rangle$. In this case, $\ukest{n}{z}(x_n, x_{n + 1}) = p_\delta \langle z \rangle (x_n, y_n, x_{n + 1}, y_{n + 1})$, which yields
\begin{equation} \label{eq:ukmod:durham:gallant:case:2}
\ukmod{n}(x_n, \rmd x_{n + 1}) = \bar{p}_\precpar^{\delta / \precpar}(x_n, y_n, x_{n + 1}, y_{n + 1}) \, \rmd x_{n + 1}, 
\end{equation}
where $\bar{p}_\precpar^{\delta / \precpar}$ denotes the transition density of (the $\delta / \precpar$-skeleton) $\bar{\kernel{P}}_\precpar^{\delta / \precpar}$.  
\end{example}

\begin{example}[the exact algorithm] \label{ex:exact:algorithm}
We consider again the partially observed diffusion process model in Example~\ref{eq:durham:gallant}. In the special case where the diffusion process governed by \eqref{eq:SDE} can be transformed into one with a constant diffusion term through the \emph{Lamperti transformation}, it was shown in \cite{beskos:papaspiliopoulos:roberts:fearnhead:2006,fearnhead2008particle} how unbiased estimation of $q_\delta$ can be achieved using generalised Poisson estimators. In our setup, this simply yields $\udmod{n} = \ud{n}$ for all $n$. We refer to the mentioned papers for details. 
\end{example}

\begin{example} 
[approximate Bayesian computation smoothing]
Consider joint smoothing in a fully dominated general state-space HMM for which the state likelihood functions $(\md{n}(\cdot, y_n))_{n \in \nset}$ are intractable (or expensive to evaluate) for any given sequence $(y_n)_{n \in \nset}$ of observations in $\rset^{d_y}$. In the case where it is possible (or faster) to sample observation emissions according to the kernels $(\mk_n)_{n \in \nset}$ one may then 
take an \emph{approximate-Bayesian-computation} (ABC) approach (see {\eg}  \cite{marin:pudlo:robert:ryder:2012}), and replace any value $\md{n}(x_n, y_n)$ by a point estimate $\kappa_\precpar(\zeta_n - y_n)$, where $\zeta_n \sim \mk_n(x_n, \cdot)$ and $\kappa_\varepsilon$ is a $d_y$-dimensional kernel density scaled by some bandwidth $\precpar > 0$. In \cite{martin:jasra:singh:whiteley:delmoral:maccoy:2014}, the authors apply the forward-only smoothing approach of \cite{delmoral:doucet:singh:2010} to this approximate model, yielding a particle-based ABC smoothing algorithm. Also this framework is covered by \hypref{assum:biased:estimate}, by letting $\ukest{n}{z}(x_n, x_{n + 1}) = \hd{n}(x_n, x_{n + 1}) \kappa_\precpar(z - y_{n + 1})$ and $\ukdist{n}(x_n, x_{n + 1}, \rmd z) = \mk_n(x_{n + 1}, \rmd z)$. In this case, 
\begin{equation} \label{eq:ukmod:ABC:smoothing}
\ukmod{n}(x_n, \rmd x_{n + 1}) = \hk_n(x_n, \rmd x_{n + 1}) \int \kappa_\precpar(z - y_{n + 1}) \, \mk_n(x_{n + 1}, \rmd z). 
\end{equation}
\end{example}




\section{Theoretical results}
\label{sec:theoretical:results}
\subsection{Convergence of pseudo-marginal PaRIS estimates}
\label{sec:convergence:results}

\subsubsection{Convergence of Algorithm~\ref{alg:pm:PaRIS}}
\label{sec:convergence:pm:PaRIS}

In \cite{olsson:westerborn:2014b}, the authors established strong consistency and asymptotic normality of the PaRIS in the framework of fully dominated general state-space HMMs and the bootstrap particle filter, \ie, in the simple case where $\adjfuncforward{n} \equiv 1$ and $\kissforward{n}{n} \equiv \hd{n}$ for all $n$. In the following we will extend these results to the considerably more general setting comprising models \eqref{eq:def:post} and the pseudo-marginal PaRIS in Algorithm~\ref{alg:pm:PaRIS}. More precisely, we will show that each weighted sample $(\epart{n}{i}, \tstat[i]{n}, \ewght{n}{i})_{i = 1}^N$, $n \in \nset$, produced by Algorithm~\ref{alg:pm:PaRIS} satisfies exponential concentration (Theorem~\ref{cor:hoeffding:tau:marginal}) and asymptotic normality (Theorem~\ref{cor:clt:pseudo:marginal:paris}) with respect to the expected additive functional $h_n$ under the \emph{skew} path model 
$$
\postmod{0:n}(\rmd x_{0:n}) \propto \chi(\rmd x_0) \prod_{m = 0}^{n - 1} \ukmod{m}(x_m, \rmd x_{m + 1}), \quad n \in \nset. 
$$
Even though these results are established along the lines of the corresponding proofs in \cite{olsson:westerborn:2014b}, it is the matter of non-trivial adaptations. As explained in Section~\ref{sec:introduction}, a random-weight particle filter (iterated pseudo-marginal forward sampling) can be viewed as a standard particle filter evolving on an extended state space comprising also the states of the auxiliary variables, and hence its convergence follows from standard SMC convergence results \cite{fearnhead2008particle}. On the contrary, since Algorithm~\ref{alg:pm:PaRIS} involves \emph{two} levels of pseudo marginalisation (with respect to both the forward-sampling and the backward-sampling operations), it cannot be described equivalently as a special instance of the original PaRIS (even in its general form given by Algorithm~\ref{alg:ideal:PaRIS}) operating on an extended space. Thus, there is no free lunch when it concerns the theoretical analysis of this scheme. Furthermore, in the case of fully dominated HMMs and when forward sampling is guided by the bootstrap filter, which was the setting in \cite{olsson:westerborn:2014b}, the conditional distribution of the particles given their ancestors (\ie, the marginal of $\rho_n$ in \eqref{eq:cond:instrumental:mixture} with respect to $x$) coincides, at any time point $n$, with the denominator of the backward kernel. Mathematically, this enables a cancellation that facilitates the analysis significantly. However, this simplification is not possible once the particle dynamics is guided by general proposal kernels, which is necessarily the case in Algorithm~\ref{alg:pm:PaRIS}. This complicates the analysis; see Remark~\ref{rem:non-trivial:extension} in supplement for further details.

To be able to describe our results in full detail, we also need to introduce, for every $n \in \nset$, the skew backward Markov kernels 
$$
    \bkmod{n}(x_{n + 1}, \rmd x_n) \eqdef \frac{\postmod{n}(\rmd x_n) \, \udmod{n}(x_n, x_{n + 1})}{\postmod{n}[\udmod{n}(\cdot, x_{n + 1})]}
$$
on $\Xset_{n + 1} \times \Xfd_n$ as well as the joint law 
\begin{equation} \label{eq:skew:backward:law}
\tstatmod{n}(x_n, \rmd x_{0:n - 1}) \eqdef \prod_{m = 0}^{n - 1} \bkmod{m}(x_{m + 1}, \rmd x_m)
\end{equation}
on $\Xset_n \times \Xfd^{n - 1}$. 

The analysis will be carried through under the following assumption.
\begin{hypH}
\label{assum:bound:filter:pseudomarginal}
For all $n \in \nset$, the functions $\adjfuncforward{n}$ and  
\begin{align*}
&\wgtfuncext{n} : \Xset_n \times \Xset_{n + 1} \times \Zset_{n + 1} \ni (x_n, x_{n + 1}, z) \mapsto \frac{\ukest{n}{z}(x_n, x_{n + 1})}{\adjfuncforward{n}(x) \kissforward{n}{n}(x_n, x_{n + 1})}, \\
&\wgtfuncmod{n} : \Xset_n \times \Xset_{n + 1} \ni (x_n, x_{n + 1}) \mapsto \frac{\udmod{n}(x_n, x_{n + 1})}{\adjfuncforward{n}(x_n) \kissforward{n}{n}(x_n, x_{n + 1})}
\end{align*}
are bounded. So is also $\initwgtfunc : \Xset_0 \ni x_0 \mapsto \rmd \chi / \rmd \init(x_0)$. 
\end{hypH}

%%% Hoeffding bound 

\subsubsection*{Hoeffding inequality}

Our first theoretical result is the following Hoeffding-type concentration inequality, which also plays a critical role for the derivation of the central limit theorem (CLT) in Theorem~\ref{cor:clt:pseudo:marginal:paris}. For every $n \in \nset$, let $\bmaf{\Xfd^n}$ denote the set of additive functionals \eqref{eq:add:func} such that $\addf{m} \in \bmf{\Xfd_m \tensprod \Xfd_{m + 1}}$ for all $m \in \intvect{0}{n - 1}$.   
\begin{theorem}
\label{cor:hoeffding:tau:marginal}
Assume \hypref[assum:biased:estimate]{assum:bound:filter:pseudomarginal}. Then for every $n \in \nset$, $\precpar \in \precparsp$, $h_n \in \bmaf{\Xfd^n}$, and $\K \in \nsetpos$ there exists $(c_n, d_n) \in \rsetpos^2$ such that for all $\N \in \nsetpos$ and $\epsilon > 0$,
$$
\pP\left(\left| \sum_{i = 1}^\N \frac{\ewght{n}{i}}{\sumwght{n}} \tstat[i]{n} - \postmod{0:n} h_n \right| \geq \epsilon \right) \leq c_n \exp \left( - d_n \N \epsilon^2 \right).
$$
\end{theorem}

The proof of Proposition~\ref{cor:hoeffding:tau:marginal}, which is an adaptation of the proof of \cite[Theorem~1]{olsson:westerborn:2014b}, is presented in Section~\ref{sec:proof:prop:hoeffding:tau:marginal} in the supplementary paper. Since we proceed by induction and the objective functions $(h_n)_{n \in \nset}$ are additive, it is, following \cite{olsson:westerborn:2014b}, necessary to establish the result for estimators in the form $\sum_{i = 1}^\N \ewght{n}{i} \{ f_n(\epart{n}{i}) \tstat[i]{n} + \ftd{n}(\epart{n}{i}) \} / \sumwght{n}$, where $(f_n, \ftd{n}) \in \bmf{\Xfd_n}^2$. This is done in Proposition~\ref{prop:hoeffding:tau:marginal} in the supplement, and Theorem \ref{cor:hoeffding:tau:marginal} follows as a corollary of that result. 

The previous bound describes the pointwise convergence of the estimator delivered by Algorithm~\ref{alg:pm:PaRIS} as $\N$ grows for $n$ fixed, and here no attempt has been made to obtain a bound that is uniform in $n$. As we shall see shortly, in Theorem~\ref{thm:variance:bound}, the numerical stability of the algorithm can instead be established by bounding the asymptotic variance of the estimator. Finally, note that the previous bound implies that the estimator $\sum_{i = 1}^\N \ewght{n}{i} \tstat[i]{n} / \sumwght{n}$ tends $\pP$-a.s. to $\postmod{0:n} h_n$ as $\N$ tends to infinity. 

%%%% CLT 

\subsubsection*{Central limit theorem}

We now focus on the asymptotic properties of Algorithm~\ref{alg:pm:PaRIS} and furnish, in Theorem~\ref{cor:clt:pseudo:marginal:paris} below, this scheme with a CLT. On the basis of this result, the stochastic stability of the algorithm is expressed by establishing, in Theorem~\ref{thm:variance:bound}, an $\mathcal{O}(n)$ bound on the asymptotic variances. In order to be able to state these results accurately, we need some additional notation. First, let 
 for $(x_n, x_{n + 1}) \in \Xset_n \times \Xset_{n + 1}$, 
\begin{multline} \label{eq:def:ukestvar}
\ukestvar{n}(x_n, x_{n + 1}) \eqdef 
\frac{1}{\wgtfuncmod{n}(x_n, x_{n + 1})} \int \{ \wgtfunc{n}(x_n, x_{n + 1}, z) - \wgtfuncmod{n}(x_n, x_{n + 1}) \}^2 \, \ukdist{n}(x_n, x_{n + 1}, \rmd z)
\end{multline}
denote the conditional relative variance of the (random) weight $\ewght{n + 1}{i}$ given $\epart{n}{\ind{n}{i}} = x_n$ and $\epart{n + 1}{i}= x_{n + 1}$. In addition, for each $n \in \nset$ and $m \in \intvect{0}{n}$, define the kernel  
\begin{equation} \label{eq:def:retrokmodmod}
    \retrokmodmod_{m, n}(x_m', \rmd x_{0:n}) \eqdef \delta_{x_m'}(\rmd x_m) \,  
    \tstatmod{m}(x_m, \rmd x_{0:m - 1})
    \prod_{\ell = m}^{n - 1} \ukmod{\ell}(x_\ell, \rmd x_{\ell + 1})
\end{equation}
on $\Xset_m \times \Xfd^n$ as well as the centered version 
\begin{equation} \label{eq:def:retrokmodmodnorm}
\retrokmodmodnorm_{m, n} h (x_m) \eqdef  \retrokmodmod_{m, n}(h - \postmod{0:n} h)(x_m) 
\end{equation}
on the same space. 

The following is the main result of this section. 
\begin{theorem} \label{cor:clt:pseudo:marginal:paris}
Assume~\hypref[assum:biased:estimate]{assum:bound:filter:pseudomarginal}. Then for all $n \in \nset$, $\precpar \in \precparsp$, $\K \in \nsetpos$, and $h_n \in \bmaf{\Xfd^n}$, 
$$
 \sqrt{\N} \left( \sum_{i = 1}^\N \frac{\ewght{n}{i}}{\sumwght{n}} \tstat[i]{n} - \postmod{0:n} h_n  \right) 
  \dlim \sigma_n(h_n) Z, 
$$
where $Z$ is standard normally distributed and 
\begin{equation} \label{eq:as:var:decomp}
\sigma^2_n (h_n) \eqdef \frac{\chi(\initwgtfunc \retrokmodmodnorm_{0, n} h_n)^2}{(\chi \ukmod{0, n - 1} \1_{\Xset^n})^2} + \sigma^2_n \langle (\wgtfuncmod{\ell})_{\ell = 0}^{n - 1} \rangle (h_n) + \sigma^2_n \langle (\ukestvar{\ell})_{\ell = 0}^{n - 1} \rangle (h_n) 
\end{equation}
and
\begin{multline} \label{eq:non-recursive:as:var}
\sigma^2_n \langle (\varphi_\ell)_{\ell = 0}^{n - 1} \rangle (h_n) 
\eqdef \sum_{m = 0}^{n - 1} \frac{\postmod{m} \adjfuncforward{m} \postmod{m} \ukmod{m}(\varphi_m [\retrokmodmodnorm_{m + 1, n} h_n ]^2)}{(\postmod{m} \ukmod{m, n - 1} \1_{\Xset^n})^2}\\
+ \sum_{m = 0}^{n - 1} \sum_{\ell = 0}^m \frac{\postmod{m} \adjfuncforward{m} \postmod{\ell} \ukmod{\ell} \{\bkmod{\ell}(\tstatmod{\ell} h_{\ell} + \addf{\ell} - \tstatmod{\ell + 1} h_{\ell +1})^2 \ukmod{\ell + 1, m}( \bkmod{m} \varphi_m [\ukmod{m + 1, n - 1} \1_{\Xset^n}]^2
)\}}{\K^{m - \ell + 1} (\postmod{\ell} \ukmod{\ell, m - 1} \1_{\Xset^m})(\postmod{m} \ukmod{m, n - 1} \1_{\Xset^n})^2}
\end{multline}
for any sequence $(\varphi_\ell)_{\ell \in \nset}$ of measurable functions $\varphi_\ell : \Xset_\ell \times \Xset_{\ell + 1} \to \rsetnn$. 
\end{theorem}

\begin{remark}
The first term of \eqref{eq:as:var:decomp} corresponds to the contribution of the initialisation step to the asymptotic variance. This term is incorrectly missing in the asymptotic-variance expressions given in \cite[Theorem~3 and Corollary~5]{olsson:westerborn:2017}. The last term corresponds to the additional variance induced by the estimation of $(\ud{n})_{n \in \nset}$ regulated by \hypref{assum:biased:estimate}. Finally, as we shall see in Section~\ref{sec:implied:convergence:results}, the first two terms correspond jointly to the variance of the ideal PaRIS in Algorithm~\ref{alg:ideal:PaRIS}, for which $(\ud{n})_{n \in \nset}$ are assumed to be known and tractable. 
\end{remark}

Theorem~\ref{cor:clt:pseudo:marginal:paris} is established in Section~\ref{sec:proof:prop:clt:pseudo:marginal:paris} in the supplementary paper. The structure of the proof is adopted from \cite{olsson:westerborn:2017} (however, we remark again that it is, as explained above, the matter of a non-trivial extension), and the CLT in Theorem~\ref{cor:clt:pseudo:marginal:paris} is obtained directly as a corollary of a more general CLT for estimators of form $\sum_{i = 1}^\N \ewght{n}{i} \{ f_n(\epart{n}{i}) \tstat[i]{n} + \ftd{n}(\epart{n}{i}) \} / \sumwght{n}$, where $(f_n, \ftd{n}) \in \bmf{\Xfd_n}^2$; see Theorem~\ref{prop:clt:pseudo:marginal:paris} in the supplement.  

\subsubsection{Convergence of Algorithm~\ref{alg:pm:SMC}} 
\label{sec:implied:convergence:results}

Our analysis provides, as a by-product, also a CLT for the random-weight particle filter obtained by iterating the forward-sampling operation in Algorithm~\ref{alg:pm:SMC}; indeed, this result, which is similar to \cite[Theorem~3]{fearnhead2008particle}, follows immediately by letting $f_n \equiv 0$ in Theorem~\ref{prop:clt:pseudo:marginal:paris} in the supplement, and we state it below for completeness. 

\begin{proposition} \label{prop:clt:pseudo:marginal:filter}
Assume~\hypref[assum:biased:estimate]{assum:bound:filter:pseudomarginal} and let $(\epart{n}{i}, \ewght{n}{i})_{i = 1}^\N$, $n \in \nset$, be generated by Algorithm~\ref{alg:pm:SMC}. Then for all $n \in \nset$, $\precpar \in \precparsp$, and $h \in \bmf{\Xfd_n}$, 
$$
 \sqrt{\N} \left( \sum_{i = 1}^\N \frac{\ewght{n}{i}}{\sumwght{n}} h(\epart{n}{i}) - \postmod{n} h \right) 
  \dlim \tilde{\sigma}_n(h) Z, 
$$
where $Z$ is standard normally distributed and
\begin{equation} \label{eq:as:var:pm:filter}
\tilde{\sigma}^2_n(h) \eqdef \frac{\chi\{ \initwgtfunc \ukmod{0} \cdots \ukmod{n - 1}(h - \postmod{n} h) \}^2}{(\chi \ukmod{0} \cdots \ukmod{n - 1} \1_{\Xset_n})^2} + \tilde{\sigma}^2_n \langle (\wgtfuncmod{\ell})_{\ell = 0}^{n - 1} \rangle (h) + \tilde{\sigma}^2_n \langle (\ukestvar{\ell})_{\ell = 0}^{n - 1} \rangle (h) 
\end{equation}
and
\begin{equation} \label{eq:non-recursive:filter:as:var}
\tilde{\sigma}^2_n \langle (\varphi_\ell)_{\ell = 0}^{n - 1} \rangle (h) 
\eqdef \sum_{m = 0}^{n - 1} \frac{\postmod{m} \adjfuncforward{m} \postmod{m} \ukmod{m}(\varphi_m [\ukmod{m + 1} \cdots \ukmod{n - 1} (h - \postmod{n} h)
]^2)}{(\postmod{m} \ukmod{m} \cdots \ukmod{n - 1} \1_{\Xset_n})^2} 
\end{equation}
for any sequence $(\varphi_\ell)_{\ell \in \nset}$ of measurable functions $\varphi_\ell : \Xset_\ell \times \Xset_{\ell + 1} \to \rsetnn$. 
\end{proposition}

\subsubsection{Convergence of Algorithm~\ref{alg:ideal:PaRIS}}

Importantly, Theorem~\ref{cor:clt:pseudo:marginal:paris} provides, as another by-product, also a CLT for the ideal PaRIS in Algorithm~\ref{alg:ideal:PaRIS}. Indeed, in the case where every $\ud{n}$ is tractable, we may set 
\begin{equation} \label{eq:ideal:case}
\ukest{n}{z}(x_n, x_{n + 1}) = \ud{n}(x_n, x_{n + 1})
\end{equation} 
for all $z$ and define $\ukdist{n}$ arbitrarily. This implies that the relative variance \eqref{eq:def:ukestvar} is identically zero, which eliminates the last term of \eqref{eq:as:var:decomp}. Thus, in this case the asymptotic variance is given by the first two terms of \eqref{eq:as:var:decomp}, but now induced by the original dynamics $(\uk{n})_{n \in \nset}$ (as \eqref{eq:ideal:case} implies that also $\udmod{n} = \ud{n}$). This result is formulated in the following corollary, where we have defined, for each $n \in \nset$ and $m \in \intvect{0}{n}$, the kernel  
\begin{equation} \label{eq:def:retrok}
    \retrok_{m, n}(x_m', \rmd x_{0:n}) \eqdef \delta_{x_m'}(\rmd x_m) \,  
    \tstat{n}(x_m, \rmd x_{0:m - 1})
    \prod_{\ell = m}^{n - 1} \uk{\ell}(x_\ell, \rmd x_{\ell + 1})
\end{equation}
on $\Xset_m \times \Xfd^n$ as well as the centered version 
$$
\retroknorm_{m, n} h (x_m) \eqdef  \retrok_{m, n}(h - \post{0:n} h)(x_m) 
$$
on the same space. 

\begin{corollary} \label{cor:clt:ideal:paris}
For each $n \in \nset$, assume that the functions $\adjfuncforward{n}$ and  
$$
\wgtfuncideal{n} : \Xset_n \times \Xset_{n + 1} \ni (x_n, x_{n + 1}) \mapsto \frac{\ud{n}(x_n, x_{n + 1})}{\adjfuncforward{n}(x) \kissforward{n}{n}(x_n, x_{n + 1})} 
$$
are bounded and let $(\epart{n}{i}, \tstat[i]{n}, \ewght{n}{i})_{i = 1}^\N$, $n \in \nset$, be generated by Algorithm~\ref{alg:ideal:PaRIS}. Then for all $n \in \nset$, $\precpar \in \precparsp$, $\K \in \nsetpos$, and $h_n \in \bmaf{\Xfd^n}$, 
$$
\sqrt{\N} \left( \sum_{i = 1}^\N \frac{\ewght{n}{i}}{\sumwght{n}} \tstat[i]{n} - \post{0:n} h_n  \right) \dlim \bar{\sigma}_n (h_n) Z, 
$$
where $Z$ is standard normally distributed and
\begin{multline} \label{eq:as:var:ideal:PaRIS}
\bar{\sigma}^2_n (h_n) 
\eqdef \frac{\chi(\initwgtfunc \retroknorm_{0, n} h_n)^2}{(\chi \ukmod{0, n - 1} \1_{\Xset^n})^2} + \sum_{m = 0}^{n - 1} \frac{\post{m} \adjfuncforward{m} \post{m} \uk{m}(\wgtfuncideal{m} [\retroknorm_{m + 1, n} h_n ]^2)}{(\post{m} \uk{m, n - 1} \1_{\Xset_n})^2} \\
+ \sum_{m = 0}^{n - 1} \sum_{\ell = 0}^m \frac{\post{m} \adjfuncforward{m} \post{\ell} \uk{\ell} \{\bkw{\ell}(\tstat{\ell} h_{\ell} + \addf{\ell} - \tstat{\ell + 1} h_{\ell +1})^2 \uk{\ell + 1, m}( \bkw{m} \wgtfuncideal{m} [\uk{m + 1, n - 1} \1_{\Xset^n}]^2
)\}}{\K^{m - \ell + 1} (\post{\ell} \uk{\ell, m - 1} \1_{\Xset_m})(\post{m} \uk{m, n - 1} \1_{\Xset_n})^2}. 
\end{multline}
\end{corollary}

Corollary~\ref{cor:clt:ideal:paris} extends \cite[Corollary~5]{olsson:westerborn:2017} to general models \eqref{eq:def:post} and auxiliary-particle-filter-guided forward sampling. 

\subsection{Long-term stochastic stability}

In this section we establish the long-term stochastic stability of Algorithm~\ref{alg:pm:PaRIS} by providing an $\mathcal{O}(n)$ bound on the sequence $(\sigma^2_n (h_n))_{n \in \nset}$ for $\K \geq 2$. Using $\K \geq 2$ is critical, since, as noted in \cite{olsson:westerborn:2014b}, using $\K = 1$ in the PaRIS leads to a path-degeneracy phenomenon similar to that of the naive smoother described in Section~\ref{sec:SMC}; we refer to \cite[Section~3.1]{olsson:westerborn:2014b} for a detailed discussion. Still, the variance bounds that we will present are of order $n(1 + 1/(\K - 1))$, which means that large $\K$ do not serve to reduce the variance significantly. This is in good agreement with simulations, where $\K = 2$ leads generally to good results. In addition, we shall see that our analysis, which is carried through in detail in Section~\ref{sec:variance:bounds} in the supplementary paper, yields, as by-products, a similar bound for the ideal PaRIS in Algorithm~\ref{alg:ideal:PaRIS} as well as a time-uniform bound on the sequence $(\tilde{\sigma}^2(h))_{n \in \nset}$ of asymptotic variances of the random-weight-particle-filter estimators generated by Algorithm~\ref{alg:pm:SMC}. As far as we know, the latter result is the first of its kind. The analysis will be carried through under the following strong-mixing assumption, which is now classical (see, \eg, \cite[Chapter~4]{delmoral:2004} and \cite[Section~4.3]{Cappe:2005:IHM:1088883}) and typically requires the state spaces $(\Xset_n)_{n \in \nset}$ to be compact sets. 
\begin{hypH}
\label{assum:strong:mixing}
There exist constants $0 < \udlow < \udup < \infty$ such that for all $m \in \nset$ and $(x_m, x_{m + 1}) \in \Xset_m \times \Xset_{m + 1}$, 
$$
    \udlow \leq \ud{m}(x_m, x_{m + 1}) \leq \udup
$$ 
and 
$$
    \udlow \leq \inf_{\precpar \in \precparsp} \udmod{m}(x_m, x_{m + 1}), \quad \sup_{\precpar \in \precparsp} \udmod{m}(x_m, x_{m + 1}) \leq \udup. 
$$ 
\end{hypH}
Note that under \hypref{assum:strong:mixing}, each reference measure $\mu_m$ is finite; we may hence, without loss of generality, assume that each $\mu_m$ is a probability measure. Under \hypref{assum:strong:mixing}, define the mixing rate 
\begin{equation} \label{eq:def:rho}
    \rho \eqdef 1 - \frac{\udlow}{\udup}
\end{equation}
as well as the constants 
$$
c(\sigma_\pm) \eqdef \frac{1}{\rho^2 (1 - \rho)^5 \udlow}, \quad d(\sigma_\pm) \eqdef \frac{1}{(1- \rho)^4 \udlow^2} \left( 2 + \frac{1}{1 - \rho} \right)^2, 
$$
and 
$$
\cboundtd \eqdef \frac{1}{(1 - \rho)^3 \udlow} \left( \frac{1}{\rho^2(1 - \rho^2)} + 1 \right). 
$$

Having introduced these quantities, we are ready to present the main result of this section. 

\begin{theorem} \label{thm:variance:bound}
Assume \hypref[assum:biased:estimate]{assum:strong:mixing}. 
Then for every $\precpar \in \precparsp$, $\K \geq 2$, $h_n \in \bmaf{\Xfd^n}$, and sequence $(\varphi_\ell)_{\ell \in \nset}$ of bounded measurable functions $\varphi_\ell : \Xset_\ell \times \Xset_{\ell + 1} \to \rsetnn$, 
$$
\limsup_{n \to \infty} \frac{1}{n} \sigma^2_n \langle (\varphi_\ell)_{\ell = 0}^{n - 1} \rangle (h_n) 
\leq \left( \cbound + \frac{1}{\K - 1} \dbound \right) 
\sup_{\ell \in \nset} \| \addf{\ell} \|_\infty^2 \sup_{\ell \in \nset} \| \adjfuncforward{\ell} \|_\infty \sup_{\ell \in \nset} \| \varphi_\ell \|_\infty.  
$$
\end{theorem}

Using Theorem~\ref{thm:variance:bound}, $\mathcal{O}(n)$ bounds on the asymptotic variances of Algorithm~\ref{alg:pm:PaRIS} and Algorithm~\ref{alg:ideal:PaRIS} are readily obtained. 

\begin{corollary} \label{cor:variance:bound}
Assume \hypref[assum:biased:estimate]{assum:strong:mixing}. Then for every $\precpar \in \precparsp$, $\K \geq 2$, and $h_n \in \bmaf{\Xfd^n}$, 
\begin{multline} \label{eq:variance:bound:full:monty}
\limsup_{n \to \infty} \frac{1}{n} \sigma^2_n(h_n) \\ 
\leq \left( \cbound + \frac{1}{\K - 1} \dbound \right) \sup_{\ell \in \nset} \| \addf{\ell} \|_\infty^2 \sup_{\ell \in \nset} \| \adjfuncforward{\ell} \|_\infty  \left( \sup_{\ell \in \nset} \| \wgtfuncmod{\ell} \|_\infty + \sup_{\ell \in \nset} \| \ukestvar{\ell} \|_\infty \right) \\
+ \frac{1}{\rho^2(1 - \rho)^4 (\chi \1_{\Xset_0})^2} \sup_{\ell \in \nset} \| \addf{\ell} \|_\infty^2. 
\end{multline}
\end{corollary}

\begin{corollary} \label{cor:variance:bound:ideal:PaRIS}
Assume \hypref[assum:biased:estimate]{assum:strong:mixing}. Then for every $\K \geq 2$ and $h_n \in \bmaf{\Xfd^n}$, 
\begin{multline*}
\limsup_{n \to \infty} \frac{1}{n} \bar{\sigma}^2_n(h_n) \leq  \left( \cbound + \frac{1}{\K - 1} \dbound \right)
\sup_{\ell \in \nset} \| \addf{\ell} \|_\infty^2 \sup_{\ell \in \nset} \| \adjfuncforward{\ell} \|_\infty \sup_{\ell \in \nset} \| \wgtfuncideal{\ell} \|_\infty \\
+ \frac{1}{\rho^2(1 - \rho)^4 (\chi \1_{\Xset_0})^2} \sup_{\ell \in \nset} \| \addf{\ell} \|_\infty^2,  
\end{multline*}
where $(\bar{\sigma}^2_n)_{n \in \nset}$ is given by \eqref{eq:as:var:ideal:PaRIS}. 
\end{corollary}

Finally, we provide, for completeness, a time-uniform bound on the asymptotic variances of the random-weight particle filter corresponding to repeated forward sampling (Algorithm~\ref{alg:pm:SMC}). This bound is obtained more or less for free while establishing Theorem~\ref{thm:variance:bound} (see Section~\ref{sec:variance:bounds} in the supplement for details).  

\begin{proposition}  \label{prop:variance:bound:filter}
Assume \hypref[assum:biased:estimate]{assum:bound:filter:pseudomarginal} and \hypref{assum:strong:mixing}. Then for every $\precpar \in \precparsp$ and $h \in \bmf{\Xfd_n}$,  
$$
\tilde{\sigma}^2_n(h) 
\leq \cboundtd  \|h \|_\infty^2 \sup_{\ell \in \nset} \| \adjfuncforward{\ell} \|_\infty  \left( \sup_{\ell \in \nset} \| \wgtfuncmod{\ell} \|_\infty + \sup_{\ell \in \nset} \| \ukestvar{\ell} \|_\infty \right) \\
+ \| h \|_\infty^2 \frac{\rho^{2n}}{\rho^4(1 - \rho)^2 (\chi \1_{\Xset_0})^2},   
$$
where $(\tilde{\sigma}^2_n)_{n \in \nset}$ are given by \eqref{eq:as:var:pm:filter}. 
\end{proposition}

\subsection{Bounds on asymptotic bias}
\label{sec:lipschitz:continuity}

In the previous section we saw that asymptotically, as $\N$ tends to infinity, the estimator produced by $n$ iterations of Algorithm~\ref{alg:pm:PaRIS} converges to the `skew' expectation $\postmod{0:n} h_n$. In this part we will study the discrepancy between $\postmod{0:n} h_n$ and $\post{0:n} h_n$ and establish an $\mathcal{O}(n \precpar)$ bound on the same. The analysis will be performed under the assumption that the precision parameter $\precpar$ controls, uniformly in $n$, the bias of the estimators provided by \hypref{assum:biased:estimate} in the following sense.    

\begin{hypH}
\label{assum:bias:bound}
There exists a constant $c > 0$ such that for all $n \in \mathbb{N}$, $\varepsilon \in \precparsp$, $h \in \bmf{\Xfd_n \tensprod \Xfd_{n + 1}}$, and $x \in \Xset_n$,   
$$
|\ukmod{n} h (x) - \uk{n} h (x)| \leq c \varepsilon \| h \|_\infty. 
$$
\end{hypH}

\begin{example}[Durham--Gallant estimator, cont.]
We check \hypref{assum:bias:bound} for the Durham--Gallant estimator in Example~\ref{eq:durham:gallant}. 

\subsubsection*{Case~1}
Assume that the emission densities of the model are uniformly bounded, \ie, there exists $\sigma_+ \in \rsetpos$ such that $\| \md{n} \|_\infty \leq \sigma_+$ for all $n \in \nset$. In the case of the Euler scheme and under the given assumptions on equation \eqref{eq:SDE}, it can be shown that there exist $c_\delta > 0$ and $d_\delta > 0$ such that for all $\precpar \in \precparsp_\delta$ and $(x_n, x_{n + 1}) \in \Xset^2$, 
\begin{equation} \label{eq:Euler:bound}
\left| q_\delta(x_n, x_{n + 1}) - \int \bar{\kernel{Q}}^{\delta / \precpar - 1}_\precpar(x_n, \rmd x) \, \bar{q}_\precpar(x, x_{n + 1}) \right|\leq c_\delta \frac{\precpar}{\delta} \exp \left( - d_\delta \| x_{n + 1} - x_n \|^2\right);  
\end{equation}
see \cite{bally:talay:1996} (see also \cite{delmoral:jacod:2001} for an application to SMC methods). Thus, using \eqref{eq:ukmod:durham:gallant}, for all $x_n$ and $h \in \bmf{\Xfd^{\tensprod 2}}$,
\begin{align*}
\left|\ukmod{n} h (x_n) - \uk{n} h (x_n) \right|&\leq c_\delta \frac{\precpar}{\delta} \int h(x_n, x_{n + 1}) g(x_n, x_{n + 1}, y_{n + 1}) \exp \left( - d_\delta \| x_{n + 1} - x_n \|^2\right) \, \rmd x_{n + 1} \\
&\leq c_\delta \frac{\precpar}{\delta} \left( \frac{\pi}{d_\delta} \right)^{d_x / 2} \sigma_+ \|h \|_\infty.  
\end{align*}

\subsubsection*{Case~2}

The second case can be treated straightforwardly by combining the bound \eqref{eq:Euler:bound}, applied to the bivariate process $(X_t, Y_t)_{t > 0}$, with \eqref{eq:ukmod:durham:gallant:case:2}. This provides the existence of constants $\tilde{c}_\delta > 0$ and $\tilde{d}_\delta > 0$ such that for all $\precpar \in \precparsp_\delta$ and $(x_n, y_n, y_{n + 1}) \in \Xset \times \Yset^2$ and $h \in \bmf{\Xfd^{\tensprod 2}}$,  
\begin{align*}
\lefteqn{\left|\ukmod{n} h (x_n) - \uk{n} h (x_n) \right|}\hspace{20mm} \\
&\leq \tilde{c}_\delta \frac{\precpar}{\delta} \exp \left( - \tilde{d}_\delta \| y_{n + 1} -y_n \|^2\right) \int h(x_n, x_{n + 1}) \exp \left( - \tilde{d}_\delta \| x_{n + 1} - x_n \|^2\right) \, \rmd x_{n + 1} \\
&\leq \tilde{c}_\delta \frac{\precpar}{\delta} \left( \frac{\pi}{\tilde{d}_\delta} \right)^{d_x / 2} \|h \|_\infty.  
\end{align*}

\end{example}

\begin{example}[the exact algorithm, cont.] \label{ex:exact:algorithm:cont}
In this case, the estimator is unbiased; thus, \hypref{assum:bias:bound} holds true for $\precpar = 0$. 
\end{example}

\begin{example}[ABC smoothing, cont.] In \cite{martin:jasra:singh:whiteley:delmoral:maccoy:2014}, the authors carry through their theoretical analysis under the assumption that each emission density $\md{n}$ is Lipschitz in the sense that there exists some constant $L \in \rsetpos$ such that   
\begin{equation} \label{eq:lipschitz:md}
\sup_{x_n \in \Xset_n} \left| \md{n}(x_n, y_n) - \md{n}(x_n, y'_n) \right| \leq L \| y_n - y_n' \|_1
\end{equation}
for all $y_n$ and $y'_n$ in $\rset^{d_y}$. For the purpose of illustration, assume that $\kappa_\precpar$ is a zero-mean multivariate normal distribution with covariance matrix $\precpar^2 \mathbf{I}_{d_y}$ for $\precpar > 0$. It is then easily shown that the condition \eqref{eq:lipschitz:md} implies \hypref{assum:bias:bound}; indeed, in this case, for all $x_n \in \Xset_n$ and $h \in \bmf{\Xfd_n \tensprod \Xfd_{n + 1}}$, using \eqref{eq:ukmod:ABC:smoothing}, 
$$
\left| \ukmod{n} h (x_n) - \mathbf{L}_n h (x_n) \right| \leq L  \|h \|_\infty \int \kappa_\precpar(z - y_{n + 1}) \| z - y_{n + 1} \|_1 \, \rmd z \leq \precpar d_y L \|h \|_\infty. 
$$
\end{example}

Under \hypref{assum:strong:mixing} and \hypref{assum:bias:bound} we may establish the next theorem, whose proof is postponed to Section~\ref{sec:proofs} in the supplement.  

\begin{theorem} \label{thm:bias:bound}
    Assume \hypref{assum:biased:estimate} and \hypref[assum:strong:mixing]{assum:bias:bound}. Then for all $n \in \nset$, $\precpar \in \precparsp$, and $h_n \in \bmaf{\Xfd^n}$, 
    \begin{align*}
        \precpar^{-1} \big| \postmod{0:n} h_n -  \post{0:n} h_n \big| 
        &\leq 2 c \frac{\udup}{\udlow^2} \sum_{k = 0}^{n - 1} \| \addf{k} \|_\infty \left( \sum_{m = 1}^{n - 1} \rho^{|k - m| - 1} + 1 \right) \\
        &\leq 2 c n \frac{\udup}{\udlow^2} \left( 1 + \frac{1}{\rho} + \frac{2}{1 - \rho} \right) \sup_{k \in \intvect{0}{n - 1}} \| \addf{k} \|_\infty,   
    \end{align*}
where $c$ is the constant in \hypref{assum:bias:bound}. 
\end{theorem}
In the case where $\sup_{k \in \nset} \| \addf{k} \|_\infty < \infty$, the bound provided by Theorem \ref{thm:bias:bound} is $\mathcal{O}(n)$. Moreover, by letting $\addf{k} \equiv 0$, for $k \in \intvect{0}{n - 2}$ and $\addf{n - 1}(x_{n - 1}, x_n) = h(x_n)$ for some given objective function $h \in \bmf{\Xfd_n}$, Theorem \ref{thm:bias:bound} provides, as a by-product, the following uniform error bound for the marginals (a result referred to as the \emph{filter sensitivity} in the case of parametric state-space models).  

\begin{corollary} \label{cor:filter:sensitivity}
 Assume \hypref{assum:biased:estimate} and \hypref[assum:strong:mixing]{assum:bias:bound}. Then for all $n \in \nset$, $\precpar \in \precparsp$, and $h \in \bmaf{\Xfd^n}$,
        $$
        \precpar^{-1} \big| \postmod{n} h -  \post{n} h \big| \leq 2 c \frac{\udup}{\udlow^2}  \| h \|_\infty \left( 1 + \frac{1}{\rho (1 - \rho)} \right).   
    $$
\end{corollary}

\begin{remark}
Consider now a parameterised version of the model, where the transition densities $(\ud{n ; \theta})_{k \in \nset}$ are indexed by some parameter $\theta$ belonging to some parameter space $\Theta$ being a subset of $\rset^d$. Assume further that all $(\ud{n ; \theta})_{n \in \nset}$ are differentiable with respect to $\theta$ and such that for all $n \in \nset$, $x_n \in \Xset$, and $h \in \bmf{\Xfd_{n + 1}}$,  
$$
\nabla_\theta \int \ud{n ; \theta}(x_n, x_{n + 1}) h(x_{n + 1}) \, \mu_{n + 1}(\rmd x_{n + 1}) = \int \nabla_\theta \ud{n ; \theta}(x_n, x_{n + 1}) h(x_{n + 1}) \, \mu_{n + 1}(\rmd x_{n + 1})
$$
and 
$$
\sup_{\theta \in \Theta} \int |\nabla_\theta \ud{n ; \theta}(x_n, x_{n + 1})| \, \mu_{n + 1}(\rmd x_{n + 1}) \leq c < \infty 
$$
for some positive constant $c$, implying \hypref{assum:bias:bound} in the sense that for all $n \in \nset$, $h \in \bmf{\Xfd_{n + 1}}$, and $x_n \in \Xset$, 
$$
| \uk{n; \theta}h(x_n) - \uk{n; \theta'}h(x_n) | \leq c \| \theta - \theta' \| \| h \|_\infty
$$
(\emph{i.e.}, $\precpar = \| \theta - \theta' \|$ in this case). Finally, assume also that family satisfies \hypref{assum:strong:mixing} uniformly over the parameter space in the sense that for all $n \in \nset$, $(x_n, x_{n + 1}) \in \Xset^2$, and $\theta \in \Theta$, 
$
\udlow \leq \ud{n; \theta}(x_n, x_{n + 1}) \leq \udup. 
$  
Then Theorem~\ref{thm:bias:bound} provides a positive constant $d$ such that for all $n \in \nset$,  
$$
\big| \post{0:n ; \theta} h_n -  \post{0:n ; \theta'} h_n \big|  
\leq d n \| \theta - \theta' \| \sup_{k \in \nset} \| \addf{k} \|_\infty. 
$$
This extends previous results on the uniform continuity of the filter distribution (see, \eg, \cite{papavasiliou:2006,legland:oudjane:2004}) to smoothing of additive state functionals.  
\end{remark}



\section{Application to tangent filter approximation}
\label{sec:tangent:filter}
\noteJO{We need to find an alternative term for `tangent filter' in the Feynman-Kac setting. Check the paper by Del Moral.}

\section{Numerical results}
\label{sec:numerical:results}

%\appendix

\begin{appendix}

\section{Proofs}
\label{sec:proofs}
We preface the proof of Theorem~\ref{thm:bias:bound} by a few technical lemmas. For each $n \in \nset$ and $m \in \intvect{0}{n}$, define the kernel  
\begin{equation} \label{eq:def:retrokmod}
    \retrokmod_{m, n}(x_m', \rmd x_{0:n}) \eqdef \delta_{x_m'}(\rmd x_m) \, \tstatmod{m}(x_m, \rmd x_{0:m - 1}) \prod_{\ell = m}^{n - 1} \uk{\ell}(x_\ell, \rmd x_{\ell + 1}), 
\end{equation}
on $\Xset_m \times \Xfd^n$ (where $\tstatmod{m}$ is defined in \eqref{eq:skew:backward:law}). 

\begin{lemma} \label{lem:retro:prospective:id}
For all $m \in \intvect{0}{n}$ and $h \in \bmf{\Xfd^n}$, 
\begin{equation} \label{eq:retro:prospective:id}
\postmod{0:m} \uk{m, n - 1} h = \postmod{m} \retrokmod_{m, n} h.  
\end{equation}
\end{lemma}

\begin{proof}
The result is obtained easily by applying Lemma~\ref{lem:reversibility}(ii) to the skew model and using definition \eqref{eq:def:retrokmod}.
\end{proof}

The following probability measures play a key role in the following: for $h \in \bmf{\Xfd_m}$, 
\begin{align}
\noshift_{m, n} h &\eqdef \frac{\postmod{m} (h \times \retrokmod_{m, n} \1_{\Xset^n})}{\postmod{m} \retrokmod_{m, n} \1_{\Xset^n}}, \nonumber \\
\shiftfwd_{m, n} h &\eqdef \frac{\postmod{m} (h \times \ukmod{m} \retrokmod_{m + 1, n} \1_{\Xset^n})}{\postmod{m} \ukmod{m} \retrokmod_{m + 1, n} \1_{\Xset^n}}, \nonumber \\
\shiftbwd_{m, n} h &\eqdef \frac{\postmod{m - 1} \uk{m - 1}(h \times \retrokmod_{m, n} \1_{\Xset^n})}{\postmod{m - 1} \retrokmod_{m - 1, n} \1_{\Xset^n}}.  \nonumber 
\end{align}
In addition, for each $k \in \intvect{0}{n - 1}$, let 
\begin{equation} \label{eq:def:addf}
\addf[n]{k} : \Xset^n \ni x_{0:n} \mapsto \addf{k}(x_k, x_{k + 1}).  
\end{equation}
denote the extension of $\addf{k}$ to $\Xset^n$. 

\begin{lemma} \label{lemma:three:identities}
Let $n \in \nset$ and $m \in \intvect{1}{n}$. Then the following holds true for all $k \in \intvect{0}{n - 1}$. 
\begin{itemize}
\item[(i)]  
$
\displaystyle \noshift_{m, n} \left( \frac{\retrokmod_{m, n} \addf[n]{k}}{\retrokmod_{m, n} \1_{\Xset^n}} \right) = \frac{\postmod{m} \retrokmod_{m, n} \addf[n]{k}}{\postmod{m} \retrokmod_{m, n} \1_{\Xset^n}}$ \quad for $m \in \intvect{1}{n}$, 
\item[(ii)]  
$
\displaystyle \shiftfwd_{m, n} \left( \frac{\retrokmod_{m, n} \addf[n]{k}}{\retrokmod_{m, n} \1_{\Xset^n}} \right) = \frac{\postmod{m + 1} \retrokmod_{m + 1, n} \addf[n]{k}}{\postmod{m + 1} \retrokmod_{m + 1, n} \1_{\Xset^n}}
$ \quad for $m \in \intvect{k + 1}{n}$, and 
\item[(iii)] 
$
\displaystyle \shiftbwd_{m, n} \left( \frac{\retrokmod_{m, n} \addf[n]{k}}{\retrokmod_{m, n} \1_{\Xset^n}} \right) = \frac{\postmod{m - 1} \retrokmod_{m - 1, n} \addf[n]{k}}{\postmod{m - 1} \retrokmod_{m - 1, n} \1_{\Xset^n}}
$ \quad for all $m \in \intvect{1}{k}$. 
\end{itemize}
\end{lemma}

\begin{proof}
The identity (i) follows straightforwardly by the definition of $\noshift_{m, n}$. 

We hence turn to (ii), which is established by first noting that for all $m \in \intvect{k + 1}{n}$, 
$$
\frac{\retrokmod_{m, n} \addf[n]{k}}{\retrokmod_{m, n} \1_{\Xset^n}} 
= \bkmod[\postmod{m - 1}]{m - 1} \cdots \bkmod[\postmod{k + 1}]{k + 1} (\bkmod[\postmod{k}]{k} \addf{k}).  
$$
Now, since, by applying Lemma~\ref{lem:reversibility}(i) to the skew model,   
\begin{align}
\shiftfwd_{m, n} h 
&= \iint \frac{\postmod{m}(\rmd x_m) \, \ukmod{m}(x_m, \rmd x_{m + 1}) \, h(x_m) \retrokmod_{m + 1, n} \1_{\Xset^n}(x_{m + 1})}{\postmod{m} \ukmod{m} \retrokmod_{m + 1, n} \1_{\Xset^n}} \nonumber \\
&= \iint \frac{\postmod{m} \ukmod{m}(\rmd x_{m + 1}) \, \bkmod[\postmod{m}]{m}(x_{m + 1}, \rmd x_m) \, h(x_m) \retrokmod_{m + 1, n} \1_{\Xset^n}(x_{m + 1})}{\postmod{m} \ukmod{m} \retrokmod_{m + 1, n} \1_{\Xset^n}} \nonumber \\
&= \frac{\postmod{m + 1} (\bkmod[\postmod{m}]{m} h \times \retrokmod_{m + 1, n} \1_{\Xset^n})}{\postmod{m + 1} \retrokmod_{m + 1, n} \1_{\Xset^n}}, \nonumber 
\end{align}
we may establish the identity by proceeding like  
\begin{align*}
\lefteqn{\shiftfwd_{m, n} \left( \frac{\retrokmod_{m, n} \addf[n]{k}}{\retrokmod_{m, n} \1_{\Xset^n}} \right)} \\
&= \iint \frac{\postmod{m + 1}(\rmd x_{m + 1}) \, \bkmod[\postmod{m}]{m}(x_{m + 1}, \rmd x_m) \, \bkmod[\postmod{m - 1}]{m - 1}  
\cdots \bkmod[\postmod{k + 1}]{k + 1}(\bkmod[\postmod{k}]{k} \addf{k})(x_m) 
\retrokmod_{m + 1, n} \1_{\Xset^n}(x_{m + 1})}{\postmod{m + 1} \retrokmod_{m + 1, n} \1_{\Xset^n}} \\
&= \frac{\postmod{m + 1} \retrokmod_{m + 1, n} \addf[n]{k}}{\postmod{m + 1} \retrokmod_{m + 1, n} \1_{\Xset^n}}.  
\end{align*}

Finally, to check (iii), note that for all $m \in \intvect{1}{k}$, 
$$
\uk{m - 1} \retrokmod_{m, n} \addf[n]{k} = \retrokmod_{m - 1, n} \addf[n]{k}. 
$$
Thus, in this case 
$$
\shiftbwd_{m, n} \left( \frac{\retrokmod_{m, n} \addf[n]{k}}{\retrokmod_{m, n} \1_{\Xset^n}} \right) = \frac{\postmod{m - 1} \uk{m - 1} \retrokmod_{m, n} \addf[n]{k}}{\postmod{m - 1} \retrokmod_{m - 1, n} \1_{\Xset^n}} = \frac{\postmod{m - 1} \retrokmod_{m - 1, n} \addf[n]{k}}{\postmod{m - 1} \retrokmod_{m - 1, n} \1_{\Xset^n}}. 
$$
\end{proof}

\begin{lemma} \label{lem:geo:bound}
Assume \hypref{assum:strong:mixing}. Then for all $n \in \nset$, $m \in \intvect{0}{n}$, $k \in \intvect{0}{n - 1}$, and $(\lambda, \lambda') \in \probmeas{\Xfd_m}^2$, 
$$
\left|\frac{\lambda \retrokmod_{m, n} \addf[n]{k}}{\lambda \retrokmod_{m, n} \1_{\Xset^n}} - \frac{\lambda' \retrokmod_{m, n} \addf[n]{k}}{\lambda' \retrokmod_{m, n} \1_{\Xset^n}} \right| \leq \| \addf{k} \|_\infty \rho^{|k - m| - 1}. 
$$
\begin{proof}
First, assume that $m \leq k$; then note that for all $x_m \in \Xset_m$, 
\begin{multline} \label{eq:retrokmodnorm:vs:forward:kernel}
\frac{\retrokmod_{m, n} \addf[n]{k}(x_m)}{\retrokmod_{m, n} \1_{\Xset^n}(x_m)} - \frac{\retrokmod_{m, n} \addf[n]{k}(x_m)}{\retrokmod_{m, n} \1_{\Xset^n}(x_m)}
 = \fk{m}{n} \cdots \fk{k - 1}{n} (\fk{k}{n} \addf{k})(x_m) \\ 
- \fk{m}{n} \cdots \fk{k - 1}{n} (\fk{k}{n} \addf{k})(x_m),    
\end{multline}

where we have introduced the \emph{forward kernels}
$$
\fk{m}{n} h (x_m) \eqdef \frac{\uk{m}(h \times \retrokmod_{m  + 1, n} \1_{\Xset^n})(x_m)}{\retrokmod_{m, n} \1_{\Xset^n} (x_m)}, \quad x_m \in \Xset_m, \quad h \in \bmf{\Xfd_{m + 1}}. 
$$
Under \hypref{assum:strong:mixing}, each forward kernel satisfies a global Doeblin condition in the form of the uniform lower bound 
$$
\fk{m}{n} h (x_m) \geq \frac{\udlow}{\udup} \mu_{m, n} h,  
$$
where we have defined the probability measure 
$$
\mu_{m, n} h \eqdef \frac{\mu_{m + 1}(h \times \retrokmod_{m  + 1, n} \1_{\Xset^n})}{\mu_{m + 1} \retrokmod_{m  + 1, n}  \1_{\Xset^n}}, \quad h \in \bmf{\Xfd_{m + 1}} 
$$
(where $\mu_{m + 1}$ is the reference measure introduced in Section~\ref{sec:model}). Thus, by standard results for uniformly minorised Markov chains (see, \emph{e.g.}, \cite[Lemma~4.3.13]{Cappe:2005:IHM:1088883}), the Dobrushin coefficient of each $\fk{m}{n}$ is bounded by $\rho = 1 - \udlow / \udup$. Thus, \eqref{eq:retrokmodnorm:vs:forward:kernel} implies that 
\begin{align*}
\left| \frac{\lambda \retrokmod_{m, n} \addf[n]{k}}{\lambda \retrokmod_{m, n} \1_{\Xset^n}} - \frac{\lambda' \retrokmod_{m, n} \addf[n]{k}}{\lambda' \retrokmod_{m, n} \1_{\Xset^n}} \right| 
&= \left| (\lambda_{m, n} - \lambda_{m, n}') \fk{m}{n} \cdots \fk{k - 1}{n} (\fk{k}{n} \addf{k})
\right| \\
&\leq \rho^{k - m} \| \fk{k}{n} \addf{k} \|_\infty \leq \rho^{k - m} \| \addf{k} \|_\infty, 
\end{align*}
where for $h \in \bmf{\Xfd_m}$, 
$$
\lambda_{m, n} h \eqdef \frac{\lambda(h \times \retrokmod_{m, n} \1_{\Xset^n})}{\lambda \retrokmod_{m, n} \1_{\Xset^n}}, \quad 
\lambda_{m, n}' h \eqdef \frac{\lambda'(h \times \retrokmod_{m, n} \1_{\Xset^n})}{\lambda' \retrokmod_{m, n} \1_{\Xset^n}}. 
$$
Now, assume that $m > k$; then note that for all $x_m \in \Xset_m$, 
\begin{multline} \label{eq:retrokmodnorm:vs:backward:kernel}
\frac{\retrokmod_{m, n} \addf[n]{k}(x_m)}{\retrokmod_{m, n} \1_{\Xset^n}(x_m)} - \frac{\retrokmod_{m, n} \addf[n]{k}(x_m)}{\retrokmod_{m, n} \1_{\Xset^n}(x_m)} \\
 = \bkmod{m - 1} \cdots \bkmod{k + 1}(\bkmod{k} \addf{k})(x_m) - \bkmod{m - 1} \cdots \bkmod{k + 1}(\bkmod{k} \addf{k})(x_m).     
\end{multline}
Under \hypref{assum:strong:mixing}, also each backward kernel satisfies a Doeblin condition, namely   
$$
\bkmod{m} h (x_{m + 1}) \geq \frac{\udlow}{\udup} \postmod{m} h,  
$$
with the marginal $\postmod{m}$ playing the role as minorising measure. Thus, the backward kernel Dobrushin coefficients are bounded by the same constant $\rho$, implying, via \eqref{eq:retrokmodnorm:vs:backward:kernel}, that 
$$
\left| \frac{\lambda \retrokmod_{m, n} \addf[n]{k}}{\lambda \retrokmod_{m, n} \1_{\Xset^n}} - \frac{\lambda' \retrokmod_{m, n} \addf[n]{k}}{\lambda' \retrokmod_{m, n} \1_{\Xset^n}} \right| 
= \left| (\lambda_{m, n} - \lambda_{m, n}') \bkmod{m - 1} \cdots \bkmod{k + 1}(\bkmod{k} \addf{k}) \right| \leq \rho^{m - k - 1} \| \addf{k} \|_\infty. 
$$
This completes the proof. 
\end{proof}
\end{lemma}

\begin{lemma} \label{lem:diff:bound} 
Assume \hypref{assum:strong:mixing} and \hypref{assum:bias:bound}. Then the following holds. 
\begin{itemize}
\item[(i)] For all $n \in \nset$, $m \in \intvect{0}{n}$, $\precpar \in \precparsp$, and $h \in \bmf{\Xfd^n}$,   
$$
\left| \noshift_{m, n} h - \shiftbwd_{m, n} h \right| \vee \left| \shiftfwd_{m, n} h - \noshift_{m, n} h \right| \leq 2 c \precpar \frac{\udup}{\udlow^2} \| h \|_\infty.   
$$
\item[(ii)] For all $n \in \nset$, $m \in \intvect{0}{n - 1}$, and $\precpar \in \precparsp$,
$$
\left| \frac{\postmod{m + 1} \retrokmod_{m + 1, n} \addf[n]{m}}{\postmod{m + 1} \retrokmod_{m + 1, n} \1_{\Xset^n}} - \frac{\postmod{m} \retrokmod_{m, n} \addf[n]{m}}{\postmod{m} \retrokmod_{m, n} \1_{\Xset^n}} \right| \leq 2 c \precpar \frac{\udup}{\udlow^2} \| h_m \|_\infty. 
$$
\end{itemize}
In both cases, $c$ is the constant in \hypref{assum:bias:bound}. 
\end{lemma}

\begin{proof}
We start with (i). Combining the decomposition 
\begin{multline*}
 \noshift_{m, n} h - \shiftbwd_{m, n} h = \varphi_{m, n}^\precpar h \left( \frac{\postmod{m - 1} \uk{m - 1} \retrokmod_{m, n} \1_{\Xset^n} - \postmod{m - 1} \ukmod{m - 1} \retrokmod_{m, n} \1_{\Xset^n}}{\postmod{m - 1} \retrokmod_{m - 1, n} \1_{\Xset^n}} \right) \\
+ \frac{\postmod{m - 1} \ukmod{m - 1}(h \times \retrokmod_{m, n} \1_{\Xset^n}) - \postmod{m - 1} \uk{m - 1}(h \times \retrokmod_{m, n} \1_{\Xset^n})}{\postmod{m - 1} \retrokmod_{m - 1, n} \1_{\Xset^n}} 
\end{multline*}
with \hypref{assum:bias:bound} provides the bound 
$$
\left| \noshift_{m, n} h - \shiftbwd_{m, n} h \right| \leq 2 c \precpar \frac{\| \retrokmod_{m, n} \1_{\Xset^n} \|_\infty  \| h \|_\infty}{\postmod{m - 1} \retrokmod_{m - 1, n} \1_{\Xset^n}}
$$
(where $c$ is the constant in \hypref{assum:bias:bound}). Now, since for all $x_m \in \Xset_m$,  
\begin{equation*}
\retrokmod_{m, n} \1_{\Xset^n}(x_m) = \int \ud{m}(x_m, x_{m + 1}) \retrokmod_{m + 1, n} \1_{\Xset^n}(x_{m + 1}) \, \mu_{m + 1}(\rmd x_{m + 1}) \\
\leq \udup \mu_{m + 1} \retrokmod_{m + 1, n} \1_{\Xset^n} 
\end{equation*}
and 
\begin{align*}
\lefteqn{\postmod{m - 1} \retrokmod_{m - 1, n} \1_{\Xset^n}} \\
&= \iint \postmod{m - 1}(\rmd x_{m - 1}) \, \ud{m - 1}(x_{m - 1}, x_m) \ud{m}(x_m, x_{m + 1}) \retrokmod_{m + 1, n} \1_{\Xset^n}(x_{m + 1}) \, \mu_m \tensprod \mu_{m + 1}(\rmd x_{m:m + 1}) \\ 
&\geq \udlow^2 \mu_{m + 1} \retrokmod_{m + 1, n} \1_{\Xset^n}, 
\end{align*}
implying the bound
\begin{equation} \label{eq:retrokmod:ratio:bd}
 \frac{\| \retrokmod_{m, n} \1_{\Xset^n} \|_\infty}{\postmod{m - 1} \uk{m - 1} \retrokmod_{m, n} \1_{\Xset^n}} \leq \frac{\udup}{\udlow^2}, 
\end{equation}
we may conclude that 
$| \noshift_{m, n} h - \shiftbwd_{m, n} h| \leq 2 c \precpar \udup \| h \|_\infty / \udlow^2
$. 
Along similar lines, the second difference can be bounded by the same quantity by writing   
\begin{multline*}
\shiftfwd_{m, n} h - \noshift_{m, n} h = \shiftfwd_{m, n} h \left( \frac{\postmod{m} \uk{m} \retrokmod_{m + 1, n} \1_{\Xset^n} - \postmod{m} \ukmod{m} \retrokmod_{m + 1, n} \1_{\Xset_n}}{\postmod{m} \uk{m} \retrokmod_{m + 1, n} \1_{\Xset^n}} \right) \\
+ \frac{\postmod{m} (h \times \ukmod{m} \retrokmod_{m + 1, n} \1_{\Xset^n}) - \postmod{m} (h \times \uk{m} \retrokmod_{m + 1, n} \1_{\Xset_n})}{\postmod{m} \uk{m} \retrokmod_{m + 1, n} \1_{\Xset^n}} 
\end{multline*}
and reapplying \hypref{assum:bias:bound} and \eqref{eq:retrokmod:ratio:bd}.  

We turn to (ii). By definition \eqref{eq:def:retrokmod},  
$$
\retrokmod_{m + 1, n} \addf[n]{m}(x_{m + 1}) = \int \addf{m}(x_m, x_{m + 1}) \, \bkmod{m}(x_{m + 1}, \rmd x_m) \, \retrokmod_{m + 1, n} \1_{\Xset^n}(x_{m + 1});
$$
thus, by applying Lemma~\ref{lem:reversibility}(i) to the skew model, 
\begin{align}
\frac{\postmod{m + 1} \retrokmod_{m + 1, n} \addf[n]{m}}{\postmod{m + 1} \retrokmod_{m + 1, n} \1_{\Xset^n}} &= \iint \frac{\postmod{m} \ukmod{m}(\rmd x_{m + 1}) \, \addf{m}(x_m, x_{m + 1}) \, \bkmod{m}(x_{m + 1}, \rmd x_m) \, \retrokmod_{m + 1, n} \1_{\Xset^n}(x_{m + 1})}{\postmod{m} \ukmod{m} \retrokmod_{m + 1, n} \1_{\Xset^n}} \nonumber \\
&= \iint \frac{\postmod{m}(\rmd x_m) \, \ukmod{m}(x_m, \rmd x_{m + 1}) \, \addf{m}(x_m, x_{m + 1}) \retrokmod_{m + 1, n} \1_{\Xset^n}(x_{m + 1})}{\postmod{m} \ukmod{m} \retrokmod_{m + 1, n} \1_{\Xset^n}}. \nonumber
\end{align}
We may thus decompose the quantity under consideration as  
\begin{multline*}
\frac{\postmod{m + 1} \retrokmod_{m + 1, n} \addf[n]{m}}{\postmod{m + 1} \retrokmod_{m + 1, n} \1_{\Xset^n}} - \frac{\postmod{m} \retrokmod_{m, n} \addf[n]{m}}{\postmod{m} \retrokmod_{m, n} \1_{\Xset^n}}
= \frac{\postmod{m + 1} \retrokmod_{m + 1, n} \addf[n]{m}}{\postmod{m + 1} \retrokmod_{m + 1, n} \1_{\Xset^n}} \left( \frac{\postmod{m} \uk{m} \retrokmod_{m + 1, n} \1_{\Xset^n} - \postmod{m} \ukmod{m} \retrokmod_{m + 1, n} \1_{\Xset^n}}{\postmod{m} \retrokmod_{m, n} \1_{\Xset^n}} \right) \\
+ \int \frac{\postmod{m}(\rmd x_m) \{ \ukmod{m}(\addf{m} \retrokmod_{m + 1, n} \1_{\Xset^n})(x_m) - \uk{m} (\addf{m} \retrokmod_{m + 1, n} \1_{\Xset^n})(x_m)\}}{\postmod{m} \retrokmod_{m, n} \1_{\Xset^n}} ,  
\end{multline*}
from which (ii) follows, as before, by a combination of \hypref{assum:bias:bound} and \eqref{eq:retrokmod:ratio:bd}.  
\end{proof}

\begin{proof}[Proof of Theorem~\ref{thm:bias:bound}]
Write 
$$
\postmod{0:n} h_n - \post{0:n} h_n = \sum_{k = 0}^{n - 1} \left( \postmod{0:n} \addf[n]{k} - \post{0:n} \addf[n]{k} \right), 
$$
where each term can be decomposed according to   
\begin{equation*}
\postmod{0:n} \addf[n]{k} - \post{0:n} \addf[n]{k} = 
\sum_{m = 1}^n \left( \frac{\postmod{0:m} \uk{m, n - 1} \addf[n]{k}}{\postmod{0:m} \uk{m, n - 1} \1_{\Xset^n}} - \frac{\postmod{0:m - 1} \uk{m - 1, n - 1} \addf[n]{k}}{\postmod{0:m - 1} \uk{m - 1, n - 1} \1_{\Xset^n}} \right)  
\end{equation*}
(recall that $\postmod{0} \propto \chi$). In order to bound each term of this decomposition, write, using Lemma~\ref{lem:retro:prospective:id},  
$$
\frac{\postmod{0:m} \uk{m, n - 1} \addf[n]{k}}{\postmod{0:m} \uk{m, n - 1} \1_{\Xset^n}} - \frac{\postmod{0:m - 1} \uk{m - 1, n - 1} \addf[n]{k}}{\postmod{0:m - 1} \uk{m - 1, n - 1} \1_{\Xset^n}} 
= \frac{\postmod{m} \retrokmod_{m, n} \addf[n]{k}}{\postmod{m} \retrokmod_{m, n} \1_{\Xset^n}} - \frac{\postmod{m - 1} \retrokmod_{m - 1, n} \addf[n]{k}}{\postmod{m - 1} \retrokmod_{m - 1, n} \1_{\Xset^n}}. 
$$
Now, for all $m \in \intvect{1}{n}$, pick an arbitrary element $\xarb_m \in \Xset_m$ and define the kernel 
\begin{equation} \label{eq:def:norm:objective:func}
\retrokmodnorm_{m, n} h(x_m) \eqdef \frac{\retrokmod_{m, n} h(x_m)}{\retrokmod_{m, n} \1_{\Xset^n}(x_m)} - \frac{\retrokmod_{m, n} h (\xarb_m)}{\retrokmod_{m, n} \1_{\Xset^n} (\xarb_m)}, \quad x_m \in \Xset_m, \quad h \in \bmf{\Xfd^n}. 
\end{equation}
Combining this definition with Lemma~\ref{lemma:three:identities}, we may express the quantity of interest as 
\begin{multline*}
\postmod{0:n} \addf[n]{k} - \post{0:n} \addf[n]{k} = \sum_{m = 1}^k \left( \noshift_{m, n} \retrokmodnorm_{m, n} \addf[n]{k} - \shiftbwd_{m, n}  \retrokmodnorm_{m, n} \addf[n]{k} \right) \\ 
+ \frac{\postmod{k + 1} \retrokmod_{k + 1, n} \addf[n]{k}}{\postmod{k + 1} \retrokmod_{k + 1, n} \1_{\Xset^n}} - \frac{\postmod{k} \retrokmod_{k, n} \addf[n]{k}}{\postmod{k} \retrokmod_{k, n} \1_{\Xset^n}} + \sum_{m = k + 1}^{n - 1} \left( \shiftfwd_{m, n} \retrokmodnorm_{m, n} \addf[n]{k} - \noshift_{m, n} \retrokmodnorm_{m, n} \addf[n]{k} \right). 
\end{multline*}
Now, applying Lemmas~\ref{lem:geo:bound} and \ref{lem:diff:bound} to the previous decomposition yields
$$
\left| \postmod{0:n} \addf[n]{k} - \post{0:n} \addf[n]{k} \right| \leq 2 c \precpar \frac{\udup }{\udlow^2} \sum_{k = 0}^{n - 1} \| \addf{k} \|_\infty \left( \sum_{m = 1}^{n - 1} \rho^{|k - m| - 1} + 1 \right).  
$$
which was to be established. 

The second inequality follows straightforwardly according to 
\begin{align*}
\sum_{k = 0}^{n - 1} \| \addf{k} \|_\infty \left( \sum_{m = 1}^{n - 1} \rho^{|k - m| - 1} + 1 \right) &\leq \left( n + \frac{1}{\rho} \left( n + 2 \sum_{\ell = 1}^{n - 1} (n - \ell) \rho^\ell \right) \right) \sup_{k \in \intvect{0}{n - 1}} \| \addf{k} \|_\infty \\ 
&\leq n \left( 1 + \frac{1}{\rho} + \frac{2}{1 - \rho} \right) \sup_{k \in \intvect{0}{n - 1}} \| \addf{k} \|_\infty.  
\end{align*}

\end{proof}







\section{Technical results}
\label{sec:tech:results}
The following technical lemma is a straightforward adaption of \cite[Lemma~14]{olsson:westerborn:2014b} to the framework of Section~\ref{sec:preliminaries}. 

\begin{lemma}
\label{lem:generalized:lebesgue}
Assume \hypref[assum:biased:estimate]{assum:bound:filter:pseudomarginal} and let $\kernel{\Psi}$ be some possibly unnormalised transition kernel on $\Xset_n \times \Xfd_{n + 1}$ having transition density in $\bmf{\Xfd_n \tensprod \Xfd_{n + 1}}$ with respect to some reference measure on $\Xfd_{n + 1}$. Moreover, let $(\varphi_\N)_{\N \in \nset}$ be a sequence of functions in $\bmf{\Xfd_{n + 1}}$  for which 
\begin{itemize}
\item[(i)] there exists $\varphi \in \bmf{\Xfd_{n + 1}}$ such that for all $x \in \Xset_{n + 1}$, $\lim_{\N \to \infty} \varphi_\N(x) = \varphi(x)$ $\pP$-a.s. and 
\item[(ii)] there exists $c \in \rsetpos$ such that $\|\varphi \|_\infty \leq c$ for all $\N \in \nset$. 
\end{itemize} 
Then $\post[\N]{n} \kernel{\Psi} \varphi_\N \pplim \post{n} \kernel{\Psi} \varphi$ as $\N \to \infty$. 
\end{lemma}



\end{appendix}

\section*{Acknowledgements}

The research of J.~Olsson is supported by the Swedish Research Council, Grant~2018-05230.  

\bibliographystyle{plain}
\bibliography{glo-2020}

\end{document}


\newpage

This template helps you to create a properly formatted \LaTeXe\ manuscript.
%%%%%%%%%%%%%%%%%%%%%%%%%%%%%%%%%%%%%%%%%%%%%%
%% `\ ' is used here because TeX ignores    %%
%% spaces after text commands.              %%
%%%%%%%%%%%%%%%%%%%%%%%%%%%%%%%%%%%%%%%%%%%%%%
Prepare your paper in the same style as used in this sample .pdf file.
Try to avoid excessive use of italics and bold face.
Please do not use any \LaTeXe\ or \TeX\ commands that affect the layout
or formatting of your document (i.e., commands like \verb|\textheight|,
\verb|\textwidth|, etc.).

\section{Section headings}
Here are some sub-sections:
\subsection{A sub-section}
Regular text.
\subsubsection{A sub-sub-section}
Regular text.

\section{Text}
\subsection{Lists}

The following is an example of an \emph{itemized} list,
two levels deep.
\begin{itemize}
\item
This is the first item of an itemized list.  Each item
in the list is marked with a ``tick.''  The document
style determines what kind of tick mark is used.
\item
This is the second item of the list.  It contains another
list nested inside it.
\begin{itemize}
\item This is the first item of an itemized list that
is nested within the itemized list.
\item This is the second item of the inner list.  \LaTeX\
allows you to nest lists deeper than you really should.
\end{itemize}
This is the rest of the second item of the outer list.
\item
This is the third item of the list.
\end{itemize}

The following is an example of an \emph{enumerated} list of one level.

\begin{enumerate}
\item This is the first item of an enumerated list.
\item This is the second item of an enumerated list.
\end{enumerate}

The following is an example of an \emph{enumerated} list, two levels deep.
\begin{enumerate}
\item[1.]
This is the first item of an enumerated list.  Each item
in the list is marked with a ``tick.''  The document
style determines what kind of tick mark is used.
\item[2.]
This is the second item of the list.  It contains another
list nested inside of it.
\begin{enumerate}
\item
This is the first item of an enumerated list that
is nested within.
\item
This is the second item of the inner list.  \LaTeX\
allows you to nest lists deeper than you really should.
\end{enumerate}
This is the rest of the second item of the outer list.
\item[3.]
This is the third item of the list.
\end{enumerate}

\subsection{Punctuation}
Dashes come in three sizes: a hyphen, an intra-word dash like ``$U$-statistics'' or ``the time-homogeneous model'';
a medium dash (also called an ``en-dash'') for number ranges or between two equal entities like ``1--2'' or ``Cauchy--Schwarz inequality'';
and a punctuation dash (also called an ``em-dash'') in place of a comma, semicolon,
colon or parentheses---like this.

Generating an ellipsis \ldots\ with the right spacing
around the periods requires a special command.

\section{Fonts}
Please use text fonts in text mode, e.g.:
\begin{itemize}
\item[]\textrm{Roman}
\item[]\textit{Italic}
\item[]\textbf{Bold}
\item[]\textsc{Small Caps}
\item[]\textsf{Sans serif}
\item[]\texttt{Typewriter}
\end{itemize}
Please use mathematical fonts in mathematical mode, e.g.:
\begin{itemize}
\item[] $\mathrm{ABCabc123}$
\item[] $\mathit{ABCabc123}$
\item[] $\mathbf{ABCabc123}$
\item[] $\boldsymbol{ABCabc123\alpha\beta\gamma}$
\item[] $\mathcal{ABC}$
\item[] $\mathbb{ABC}$
\item[] $\mathsf{ABCabc123}$
\item[] $\mathtt{ABCabc123}$
\item[] $\mathfrak{ABCabc123}$
\end{itemize}
Note that \verb|\mathcal, \mathbb| belongs to capital letters-only font typefaces.

\section{Notes}
Footnotes\footnote{This is an example of a footnote.}
pose no problem.\footnote{Note that footnote number is after punctuation.}

\section{Quotations}

Text is displayed by indenting it from the left margin. There are short quotations
\begin{quote}
This is a short quotation.  It consists of a
single paragraph of text.  There is no paragraph
indentation.
\end{quote}
and longer ones.
\begin{quotation}
This is a longer quotation.  It consists of two paragraphs
of text.  The beginning of each paragraph is indicated
by an extra indentation.

This is the second paragraph of the quotation.  It is just
as dull as the first paragraph.
\end{quotation}

\section{Environments}

\subsection{Examples for \emph{\texttt{plain}}-style environments}
\begin{axiom}\label{ax1}
This is the body of Axiom \ref{ax1}.
\end{axiom}

\begin{proof}
This is the body of the proof of the axiom above.
\end{proof}

\begin{claim}\label{cl1}
This is the body of Claim \ref{cl1}. Claim \ref{cl1} is numbered after
Axiom \ref{ax1} because we used \verb|[axiom]| in \verb|\newtheorem|.
\end{claim}

\begin{theorem}\label{th1}
This is the body of Theorem \ref{th1}. Theorem \ref{th1} numbering is
dependent on section because we used \verb|[section]| after \verb|\newtheorem|.
\end{theorem}

\begin{theorem}[Title of the theorem]\label{th2}
This is the body of Theorem \ref{th2}. Theorem \ref{th2} has additional title.
\end{theorem}

\begin{lemma}\label{le1}
This is the body of Lemma \ref{le1}. Lemma \ref{le1} is numbered after
Theorem \ref{th2} because we used \verb|[theorem]| in \verb|\newtheorem|.
\end{lemma}


\begin{proof}[Proof of Lemma \ref{le1}]
This is the body of the proof of Lemma \ref{le1}.
\end{proof}

\subsection{Examples for \emph{\texttt{remark}}-style environments}
\begin{definition}\label{de1}
This is the body of Definition \ref{de1}. Definition \ref{de1} is numbered after
Lemma \ref{le1} because we used \verb|[theorem]| in \verb|\newtheorem|.
\end{definition}

\begin{example}
This is the body of the example. Example is unnumbered because we used \verb|\newtheorem*|
instead of \verb|\newtheorem|.
\end{example}

\begin{fact}
This is the body of the fact. Fact is unnumbered because we used \verb|\newtheorem*|
instead of \verb|\newtheorem|.
\end{fact}

\section{Tables and figures}
Cross-references to labeled tables: As you can see in Table~\ref{sphericcase}
and also in Table~\ref{parset}.

\begin{table*}
\caption{The spherical case ($I_1=0$, $I_2=0$)}
\label{sphericcase}
\begin{tabular}{@{}lrrrrc@{}}
\hline
Equil. \\
points & \multicolumn{1}{c}{$x$}
& \multicolumn{1}{c}{$y$} & \multicolumn{1}{c}{$z$}
& \multicolumn{1}{c}{$C$} & S \\
\hline
$L_1$    & $-$2.485252241 & 0.000000000  & 0.017100631  & 8.230711648  & U \\
$L_2$    & 0.000000000  & 0.000000000  & 3.068883732  & 0.000000000  & S \\
$L_3$    & 0.009869059  & 0.000000000  & 4.756386544  & $-$0.000057922 & U \\
$L_4$    & 0.210589855  & 0.000000000  & $-$0.007021459 & 9.440510897  & U \\
$L_5$    & 0.455926604  & 0.000000000  & $-$0.212446624 & 7.586126667  & U \\
$L_6$    & 0.667031314  & 0.000000000  & 0.529879957  & 3.497660052  & U \\
$L_7$    & 2.164386674  & 0.000000000  & $-$0.169308438 & 6.866562449  & U \\
$L_8$    & 0.560414471  & 0.421735658  & $-$0.093667445 & 9.241525367  & U \\
$L_9$    & 0.560414471  & $-$0.421735658 & $-$0.093667445 & 9.241525367  & U \\
$L_{10}$ & 1.472523232  & 1.393484549  & $-$0.083801333 & 6.733436505  & U \\
$L_{11}$ & 1.472523232  & $-$1.393484549 & $-$0.083801333 & 6.733436505  & U \\
\hline
\end{tabular}
\end{table*}

\begin{table}
\caption{Sample posterior estimates for each model}
\label{parset}
%
\begin{tabular}{@{}lcrcrrr@{}}
\hline
&& & &\multicolumn{3}{c}{Quantile} \\
\cline{5-7}
Model &Parameter &
\multicolumn{1}{c}{Mean} &
Std. dev.&
\multicolumn{1}{c}{2.5\%} &
\multicolumn{1}{c}{50\%}&
\multicolumn{1}{c@{}}{97.5\%} \\
\hline
{Model 0} & $\beta_0$ & $-$12.29 & 2.29 & $-$18.04 & $-$11.99 & $-$8.56 \\
          & $\beta_1$  & 0.10   & 0.07 & $-$0.05  & 0.10   & 0.26  \\
          & $\beta_2$   & 0.01   & 0.09 & $-$0.22  & 0.02   & 0.16  \\[6pt]
{Model 1} & $\beta_0$   & $-$4.58  & 3.04 & $-$11.00 & $-$4.44  & 1.06  \\
          & $\beta_1$   & 0.79   & 0.21 & 0.38   & 0.78   & 1.20  \\
          & $\beta_2$   & $-$0.28  & 0.10 & $-$0.48  & $-$0.28  & $-$0.07 \\[6pt]
{Model 2} & $\beta_0$   & $-$11.85 & 2.24 & $-$17.34 & $-$11.60 & $-$7.85 \\
          & $\beta_1$   & 0.73   & 0.21 & 0.32   & 0.73   & 1.16  \\
          & $\beta_2$   & $-$0.60  & 0.14 & $-$0.88  & $-$0.60  & $-$0.34 \\
          & $\beta_3$   & 0.22   & 0.17 & $-$0.10  & 0.22   & 0.55  \\
\hline
\end{tabular}
%
\end{table}

\begin{figure}
\includegraphics{figure1}
\caption{Pathway of the penicillin G biosynthesis.}
\label{penG}
\end{figure}

Sample of cross-reference to figure.
Figure~\ref{penG} shows that it is not easy to get something on paper.

\section{Equations and the like}

Two equations:
\begin{equation}
    C_{s}  =  K_{M} \frac{\mu/\mu_{x}}{1-\mu/\mu_{x}} \label{ccs}
\end{equation}
and
\begin{equation}
    G = \frac{P_{\mathrm{opt}} - P_{\mathrm{ref}}}{P_{\mathrm{ref}}}  100(\%).
\end{equation}

Equation arrays:
\begin{eqnarray}
  \frac{dS}{dt} & = & - \sigma X + s_{F} F,\\
  \frac{dX}{dt} & = &   \mu    X,\\
  \frac{dP}{dt} & = &   \pi    X - k_{h} P,\\
  \frac{dV}{dt} & = &   F.
\end{eqnarray}
One long equation:
\begin{eqnarray}
 \mu_{\text{normal}} & = & \mu_{x} \frac{C_{s}}{K_{x}C_{x}+C_{s}}  \nonumber\\
                     & = & \mu_{\text{normal}} - Y_{x/s}\bigl(1-H(C_{s})\bigr)(m_{s}+\pi /Y_{p/s})\nonumber\\
                     & = & \mu_{\text{normal}}/Y_{x/s}+ H(C_{s}) (m_{s}+ \pi /Y_{p/s}).
\end{eqnarray}
%%%%%%%%%%%%%%%%%%%%%%%%%%%%%%%%%%%%%%%%%%%%%%
%% Example with single Appendix:            %%
%%%%%%%%%%%%%%%%%%%%%%%%%%%%%%%%%%%%%%%%%%%%%%
\begin{appendix}
\section*{Title}\label{appn} %% if no title is needed, leave empty \section*{}.
Appendices should be provided in \verb|{appendix}| environment,
before Acknowledgements.

If there is only one appendix,
then please refer to it in text as \ldots\ in the \hyperref[appn]{Appendix}.
\end{appendix}
%%%%%%%%%%%%%%%%%%%%%%%%%%%%%%%%%%%%%%%%%%%%%%
%% Example with multiple Appendixes:        %%
%%%%%%%%%%%%%%%%%%%%%%%%%%%%%%%%%%%%%%%%%%%%%%
\begin{appendix}
\section{Title of the first appendix}\label{appA}
If there are more than one appendix, then please refer to it
as \ldots\ in Appendix \ref{appA}, Appendix \ref{appB}, etc.

\section{Title of the second appendix}\label{appB}
\subsection{First subsection of Appendix \protect\ref{appB}}

Use the standard \LaTeX\ commands for headings in \verb|{appendix}|.
Headings and other objects will be numbered automatically.
\begin{equation}
\mathcal{P}=(j_{k,1},j_{k,2},\dots,j_{k,m(k)}). \label{path}
\end{equation}

Sample of cross-reference to the formula (\ref{path}) in Appendix \ref{appB}.
\end{appendix}

%%%%%%%%%%%%%%%%%%%%%%%%%%%%%%%%%%%%%%%%%%%%%%
%% Support information (funding), if any,   %%
%% should be provided in the                %%
%% Acknowledgements section.                %%
%%%%%%%%%%%%%%%%%%%%%%%%%%%%%%%%%%%%%%%%%%%%%%
\section*{Acknowledgements}
The authors would like to thank the anonymous referees, an Associate
Editor and the Editor for their constructive comments that improved the
quality of this paper.

The first author was supported by NSF Grant DMS-??-??????.

The second author was supported in part by NIH Grant ???????????.

%%%%%%%%%%%%%%%%%%%%%%%%%%%%%%%%%%%%%%%%%%%%%%
%% Supplementary Material, if any, should   %%
%% be provided in {supplement} environment  %%
%% with title and short description.        %%
%%%%%%%%%%%%%%%%%%%%%%%%%%%%%%%%%%%%%%%%%%%%%%
\begin{supplement}
\stitle{Title of Supplement A}
\sdescription{Short description of Supplement A.}
\end{supplement}
\begin{supplement}
\stitle{Title of Supplement B}
\sdescription{Short description of Supplement B.}
\end{supplement}

%%%%%%%%%%%%%%%%%%%%%%%%%%%%%%%%%%%%%%%%%%%%%%%%%%%%%%%%%%%%%
%%                  The Bibliography                       %%
%%                                                         %%
%%  imsart-number.bst  will be used to                     %%
%%  create a .BBL file for submission.                     %%
%%                                                         %%
%%  Note that the displayed Bibliography will not          %%
%%  necessarily be rendered by Latex exactly as specified  %%
%%  in the online Instructions for Authors.                %%
%%                                                         %%
%%  MR numbers will be added by VTeX.                      %%
%%                                                         %%
%%  Use \cite{...} to cite references in text.             %%
%%                                                         %%
%%%%%%%%%%%%%%%%%%%%%%%%%%%%%%%%%%%%%%%%%%%%%%%%%%%%%%%%%%%%%

%% if your bibliography is in bibtex format, uncomment commands:
%\bibliographystyle{imsart-number} % Style BST file
%\bibliography{bibliography}       % Bibliography file (usually '*.bib')

%% or include bibliography directly:
%\begin{thebibliography}{4}
\bibliographystyle{plain}
\bibliography{glo-2020}

%\bibitem{r1}
%Billingsley, P. (1999). \textit{Convergence of
%Probability Measures}, 2nd ed.
%New York: Wiley.
%
%\bibitem{r2}
%Bourbaki, N.  (1966). \textit{General Topology}  \textbf{1}.
%Reading, MA: Addison--Wesley.
%
%\bibitem{r3}
%Ethier, S. N. and Kurtz, T. G. (1985).
%\textit{Markov Processes: Characterization and Convergence}.
%New York: Wiley.
%
%\bibitem{r4}
%Prokhorov, Yu. (1956).
%Convergence of random processes and limit theorems in probability
%theory. \textit{Theory  Probab.  Appl.}
%\textbf{1} 157--214.

%\end{thebibliography}

\end{document}
