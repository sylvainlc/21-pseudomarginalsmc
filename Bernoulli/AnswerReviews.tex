\documentclass[12pt]{amsart}
\usepackage[T1]{fontenc}
\usepackage[latin1]{inputenc}
\usepackage[round]{natbib}
\usepackage{amssymb,amsmath,latexsym,xcolor}

%\textwidth  14.5cm
%\textheight 20cm
%\voffset=-1cm
%\hoffset=-1cm

\newcommand{\rref}[1]{{\small \textcolor{gray}{#1}} }
\newcommand{\cl}[1]{ \textcolor{blue}{#1}}

\begin{document}


\begin{center}
\textbf{Answer to Referees for \textit{"A pseudo-marginal sequential Monte Carlo online smoothing algorithm"}}
\end{center}

\bigskip

We would like to thank the associate editor and the referees for the attention they paid to our work and for their positive feedbacks. 

\bigskip

Minor errors and typos listed by R2 were corrected as suggested. Please find below a few detailed answers.

\medskip

{\bf p.7 Just before 2.2.2, it would be nice to comment on where one might get suitable adjustment weights $\vartheta_n$ from. How are these to be chosen or tuned?}\\
{\em To complete!}

\medskip

{\bf p.29 Appendix A in the supplement has $\phi_n(dx_n)[\ell(\cdot,x_{n+1})]$  in the numerator that arises from $B_n(x_{n+1},dx_n)$. I think this should read $\phi_n(dx_n)\ell(x_n,x_{n+1})$  instead?}\\
{\em The normalizing term given in the appendix is correct, see (2.10).}

\medskip

{\bf p.30 Appendix A has $\phi_{0:n}L_{0,n}1_{X^n}$  in the denominator for  $\phi_{0:n+1}h$ which I think should read  $\phi_{0:n}L_{0,n}1_{X^{n+1}}$ as there is a dangling $dx_{n+1}$ otherwise.}\\
{\em There was indeed a mistake in this identity, thank you for pointing this out. This was corrected using (2.3).}

\medskip

\end{document}


