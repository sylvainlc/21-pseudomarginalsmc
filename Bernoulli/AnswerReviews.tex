\documentclass[12pt]{amsart}
\usepackage[T1]{fontenc}
\usepackage[latin1]{inputenc}
\usepackage[round]{natbib}
\usepackage{amssymb,amsmath,latexsym,xcolor}

%\textwidth  14.5cm
%\textheight 20cm
%\voffset=-1cm
%\hoffset=-1cm

\newcommand{\rref}[1]{{\small \textcolor{gray}{#1}} }
\newcommand{\cl}[1]{ \textcolor{blue}{#1}}

\begin{document}


\begin{center}
\textbf{Answer to Referees for \textit{"A pseudo-marginal sequential Monte Carlo online smoothing algorithm"}}
\end{center}

\bigskip

We would like to thank the associate editor and the referees for the attention they paid to our work and for their positive feedbacks. 

\bigskip

Minor errors and typos listed by R2 were corrected as suggested. Please find below a few detailed answers.


\begin{itemize}
\item p.7 (2.10) and elsewhere uses rectangular brackets as in $\phi_m[\ell_m(\cdot,x_{m+1})]$ which I think is meant to denote $\int_{X_m}\ell_m(x_m,x_{m+1})\phi_m(dx_m)$. This notation has not been introduced. The "Preliminaries" at the top of p.4 have $\mu h$ for the integral of a function $h$ against a measure $\mu$ which would suggest $\phi_m \ell(\cdot,x_{m+1})$. Since the rectangular brackets are used in this sense at many points in the paper, perhaps this notation should be added to the preliminaries.\\
{\em To do!}
\item p.7 Just before 2.2.2, it would be nice to comment on where one might get suitable adjustment weights $\vartheta_n$ from. How are these to be chosen or tuned?\\
{\em To do!}
\item p. 15-16: for the presentation of Example 2, I feel it would be better to describe the problem fully before giving Durham and Gallant's solution.\\
{\em Do not agree!}
\item p.29 Appendix A in the supplement has $\phi_n(dx_n)[\ell(\cdot,x_{n+1})]$  in the numerator that arises from $B_n(x_{n+1},dx_n)$. I think this should read $\phi_n(dx_n)\ell(x_n,x_{n+1})$  instead?\\
{\em The normalizing term given in the appendix is correct, see (2.10).}
\item p.30 Appendix A has $\phi_{0:n}L_{0,n}1_{X^n}$  in the denominator for  $\phi_{0:n+1}h$ which I think should read  $\phi_{0:n}L_{0,n}1_{X^{n+1}}$ as there is a dangling $dx_{n+1}$ otherwise.\\
{\em There was indeed a mistake in this identity, thank you for pointing this out. This was corrected using (2.3).}
\item It might have made the paper less scary and dry to read to first provide standard setups (along the lines of what is at the top of p.2), then proceed to the objective (formula (1.2)) and only then supplement all the measure-theoretic setup, e.g. as part of section 2. This would also have yielded the advantage of collecting mathematical notation at the start of section 2 (Is a collection of common notation in tabular form allowed in Bernoulli? This would be a friendly service to the reader.) rather than scattering it over the start of section 1 and section 2.\\
{\em To discuss!}
\end{itemize}

\end{document}


